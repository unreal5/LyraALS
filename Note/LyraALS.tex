\documentclass[10pt,a4paper]{colorful}

\begin{document}
\chapter{Core Collision Mechanics}

Collisions, in game development, are the first point at which differing game objects interact with each other. A collision event occurs when two or more objects touch or overlap positions in a game. Such a simple concept, yet it forms a fundamental basis for most of the gameplay mechanics---and for good reason.

Collisions are fundamental to how a player interacts with the game world. By implementing a simple collision system, we demonstrate implementation details for game interactivity. When you know this simple system, you can come up with a ton of interesting mechanics---avoiding things in the environment, interacting with objects, or even moving through a level. Imagine your player character running through a maze, avoiding walls, and picking up items. This is where our journey begins, and by the end of this chapter, you'll be able to implement these mechanics yourself.

Collisions can also be used for more advanced functionality. Advanced collision techniques are very important for high-end physics-based interactions, environmental destruction, and complex combat mechanics. Using collisions in a very clever way, we will improve gameplay incredibly, making it even more enjoyable and real. Think of games where the environment is destructible or where something bounces in a very lifelike way because of the application of advanced collision techniques. Let's take a closer look at them, and you will reach the point where you can build much more realistic and interactive games.

In this chapter, we delve into collision mechanics; we first start with the basics, then hit more complex applications. By the end, you should have a good idea of how collisions are detected and resolved and how they can be used to implement all sorts of gameplay mechanics. You will be set up to create better, more realistic games.

\section{Blocking Movement}

Blocking Movement is the mechanics that inflict limitations and restrictions on object movement. Any game simulation concerning spatial navigation must impose the laws of physics through these mechanics. Imagine you are making a maze; you wouldn't want the player to just walk through walls. Blocking movement assures that navigation of a player is realistic and fits with the physics of the in-game world.

\subsection{Basic Usage}

Blocking movement is when some object or player character can be denied free movement due to blocking obstacles. Such obstacles are talking with the moving object with areas they own. If an object or a player character enters these areas, their movement can be blocked or redirected.

This general use case will let us visualize and build the game environment by specifying positions and sizes for non-movable blockades or obstructions.

\subsection{Advanced Techniques}

These are techniques where interactive obstacles that the player can destroy or move are added, for example, a door that opens and closes after a player's action or switches to trigger environmental changes. This kind of simulation with physics could also give real collision responses as in pushing and knocking objects back.

Think of the simple addition of a door in your maze that will open and close only when a lever is pulled. It adds an interactive element, hence increasing the player's experience.

\subsection{Practical Applications}

Blocking movement plays a crucial role in game development, influencing how characters and objects interact within the game world. Below are some key applications of blocking movement:

\begin{itemize}
\item
  Creating realistic obstacles in the environment
\item
  Designing collision-based puzzles or challenges
\item
  Simulating how characters and objects interact with each other in a game
\item
  Reinforcing rules of the game world, like gravity and friction
\end{itemize}

\subsection{Algorithm Overview}

Let's imagine the algorithm for blocking movement:

\begin{enumerate}
\def\labelenumi{\arabic{enumi}.}
\item
  1.
\end{enumerate}

\begin{quote}
\textbf{Initialization}: Place the player and the obstacles in the game scene.

~
\end{quote}

\begin{enumerate}
\def\labelenumi{\arabic{enumi}.}
\setcounter{enumi}{1}
\item
  2.
\end{enumerate}

\begin{quote}
\textbf{Collision Detection}: Check repeatedly whether there is a collision between the player and the obstacles.

~
\end{quote}

\begin{enumerate}
\def\labelenumi{\arabic{enumi}.}
\setcounter{enumi}{2}
\item
  3.
\end{enumerate}

\begin{quote}
\textbf{Collision Resolution}: Detect and resolve collisions with good physics responses, including blocking and deflection of movement.

~
\end{quote}

\begin{enumerate}
\def\labelenumi{\arabic{enumi}.}
\setcounter{enumi}{3}
\item
  4.
\end{enumerate}

\begin{quote}
\textbf{Interaction Handling}: Implement interactivity within the obstacles (like doors or triggers), which modify obstacle behavior based on player actions.

~
\end{quote}

This serves as the core algorithm for blocking movement mechanics implementation within Unreal Engine.

\subsection{Reviewing the Code}

Now that we've explored the general concepts and applications of blocking movement, let's dive into the specifics of how to implement these mechanics in Unreal Engine. This section will walk you through the process of setting up your game environment, starting with the foundational elements required to create a maze and configure player movement.

\subsection{Setting Up the Maze and Player}

Having discussed general concepts and the way blocking movement is being used, now let's get into the details of actual block implementation within Unreal Engine and setting your game environment from basic components that create a maze to player movement.

\subsubsection{Configuring Maze}

Listing 1-1 declares class AMazeWall, intended to be a wall in the maze of Unreal Engine. The following is an explanation of the principal elements that perform their functions:

\begin{itemize}
\item
  \textbf{MazeWall Class}: This class inherits from AActor. This means it will become one entity that can then exist in a game world. It represents a static mesh wall inside the maze.
\item
  \textbf{BeginPlay Function}: This is a function that gets called when the game starts or when an actor is spawned. It was used to initialize any parameters or states that needed setup before the start of gameplay.
\item
  \textbf{Tick Function}: It calls every frame of gameplay. This function is used to keep updating the state of an actor continuously but doesn't change behavior directly in this case.
\item
  \textbf{MazeWall Property}: This UStaticMeshComponent will be responsible for defining the visual appearance of the wall. It is visible, and it will react to the game's physics and collision system.
\end{itemize}

\emph{\#pragma once}

\emph{class UStaticMeshComponent;}

\emph{\#include "CoreMinimal.h"}

\emph{\#include "GameFramework/Actor.h"}

\emph{\#include "MazeWall.generated.h"}

UCLASS()

\textbf{class MYGAME\_API} AMazeWall : \textbf{public} AActor

\{

~~~~GENERATED\_BODY()

public:

~~~~AMazeWall();

protected:

~~~~\textbf{virtual void} BeginPlay() override;

public:

~~~~\textbf{virtual void} Tick(\textbf{float} DeltaTime) override;

private:

~~~~UPROPERTY(VisibleAnywhere)

~~~~UStaticMeshComponent* MazeWall;

\};

\emph{\textbf{Listing 1-1}}

MazeWall.h

Listing 1-2 sets up the behavior of the class AMazeWall, through which it is defined how it will interact with other elements in the game world.

\begin{itemize}
\item
  \textbf{CreateDefaultSubobject}: A function that creates an instance of the class UStaticMeshComponent named MazeWall. This component renders the wall in the game environment and handles collision interactions.
\item
  \textbf{SetCollisionEnabled}: This method belongs to the MazeWall component and is called to activate collision detection and physics responses. This means that, when active, the wall will be able to interact with other objects and characters according to the collision system implemented in the game.
\item
  \textbf{SetCollisionResponseToAllChannels}: The ECR\_Block sets the collision response, which ensures that the wall blocks all collision channels. This simply means that the wall prevents the penetration of any player character or other objects across it, rendering it a physical barrier inside the maze.
\item
  \textbf{BeginPlay}: This is a function that gets called when the game starts or, if it's a wall actor, when the actor is spawned. It is a place to initialize any extra settings or properties; here, it just inherits the default behavior from its parent class.
\item
  \textbf{Tick}: It is called once every frame while in play. Here, this function is overridden to probably update the state of the wall or change its behavior; however, it inherits the default behavior from its parent class.
\end{itemize}

\emph{\#include "MazeWall.h"}

\#include "Components/StaticMeshComponent.h"

AMazeWall::AMazeWall()

\{

~~~~PrimaryActorTick.bCanEverTick = true;

~~~~MazeWall = CreateDefaultSubobject\textless UStaticMeshComponent\textgreater(TEXT(\emph{"MazeWall"}));

~~~~RootComponent = MazeWall;

~~~~MazeWall-\textgreater SetCollisionEnabled(ECollisionEnabled::QueryAndPhysics);

~~~~MazeWall-\textgreater SetCollisionResponseToAllChannels(ECollisionResponse::ECR\_Block);

\}

\textbf{void} AMazeWall::BeginPlay()

\{

~~~~Super::BeginPlay();

\}

\textbf{void} AMazeWall::Tick(\textbf{float} DeltaTime)

\{

~~~~Super::Tick(DeltaTime);

\}

\emph{\textbf{Listing 1-2}}

MazeWall.cpp

\subsubsection{Configuring Player Collision}

Now it's time to set up collision settings for the player character to collide with maze walls.

Listing 1-3 defines the player character in Unreal Engine. Here's a breakdown of its components and functions:

\begin{itemize}
\item
  \textbf{PlayerCharacter Class}: This class defines an actor that represents the player character in the game. It inherits from ACharacter, which provides basic functionality for a player-controlled character.
\item
  \textbf{BeginPlay Function}: This function is called when the game starts or when the player character is spawned. It is used for initialization tasks that need to occur when the actor begins play.
\item
  \textbf{Tick Function}: The Tick function is called every frame during gameplay. It is used to update the actor's state, though, in this example, it inherits default behavior from its parent class.
\item
  \textbf{PlayerCapsule}: This private member is a pointer to a UCapsuleComponent, which is used to define the collision shape of the player character. The capsule shape is commonly used for player characters to handle collisions with the environment.
\end{itemize}

\emph{\#pragma once}

class UCapsuleComponent;

\emph{\#include "CoreMinimal.h"}

\emph{\#include "GameFramework/Character.h"}

\emph{\#include "PlayerCharacter.generated.h"}

UCLASS()

\textbf{class MYGAME\_API} APlayerCharacter : \textbf{public} ACharacter

\{

~~~~GENERATED\_BODY()

public:

~~~~APlayerCharacter();

protected:

~~~~\textbf{virtual void} BeginPlay() override;

public:

~~~~\textbf{virtual void} Tick(\textbf{float} DeltaTime) override;

private:

~~~~UPROPERTY(VisibleAnywhere)

~~~~UCapsuleComponent* PlayerCapsule;

\};

\emph{\textbf{Listing 1-3}}

PlayerCharacter.h

The implementation details for the APlayerCharacter class in Listing 1-4 are as follows:

\begin{itemize}
\item
  \textbf{CreateDefaultSubobject}: This function creates an instance of UCapsuleComponent named PlayerCapsule. It is used to define the player's collision shape and is set as the root component of the player character.
\item
  \textbf{SetCollisionEnabled}: This method enables collision detection and physics responses for the PlayerCapsule component. It allows the player character to interact with the game world based on physics simulations.
\item
  \textbf{SetCollisionResponseToAllChannels}: By setting the collision response to ECR\_Block, the player character will block all collision channels. This ensures that the player character cannot pass through obstacles, such as maze walls.
\end{itemize}

\emph{\#include "PlayerCharacter.h"}

\#include "Components/CapsuleComponent.h"

APlayerCharacter::APlayerCharacter()

\{

~~~~PrimaryActorTick.bCanEverTick = true;

~~~~PlayerCapsule = CreateDefaultSubobject\textless UCapsuleComponent\textgreater(TEXT(\emph{"PlayerCapsule"}));

~~~~RootComponent = PlayerCapsule;

~~~~PlayerCapsule-\textgreater SetCollisionEnabled(ECollisionEnabled::QueryAndPhysics);

~~~~PlayerCapsule-\textgreater SetCollisionResponseToAllChannels(ECollisionResponse::ECR\_Block);

\}

\textbf{void} APlayerCharacter::BeginPlay()

\{

~~~~Super::BeginPlay();

\}

\textbf{void} APlayerCharacter::Tick(\textbf{float} DeltaTime)

\{

~~~~Super::Tick(DeltaTime);

\}

\emph{\textbf{Listing 1-4}}

PlayerCharacter.cpp

\subsection{Implementing Movement Restrictions}

Unreal Engine handles collision responses automatically, restricting player movement when they collide with maze walls. This ensures realistic navigation through the game environment.

\subsubsection{Advanced Blocking Movement}

For more interactive experiences, such as doors and switches, we can add trigger volumes (Listings 1-5 and 1-6). The ADoorTrigger class is designed to handle interactions with a trigger volume that can activate or deactivate a door or similar mechanism. Here's a breakdown of its components and functions:

\begin{itemize}
\item
  \textbf{DoorTrigger Class}: This class represents an actor that utilizes a trigger volume to detect when other actors enter or exit its defined area. This is typically used to create interactive elements, like doors, that respond to player actions.
\item
  \textbf{BeginPlay Function}: This function is called when the game starts or when the ADoorTrigger actor is spawned. It's used to set up any necessary initial states or configurations for the actor.
\item
  \textbf{Tick Function}: This function is called every frame during gameplay. It's used to update the actor's state if needed, although, in this case, it may inherit default behavior or be left unimplemented.
\item
  \textbf{DoorTrigger}: This private member is a pointer to an ATriggerVolume that defines the area where interactions occur. This trigger volume detects when other actors enter or exit its bounds.
\item
  \textbf{OnOverlapBegin Function}: This function is a callback that is triggered when another actor begins to overlap with the trigger volume. It can be used to implement logic for opening the door or activating other interactive elements.
\item
  \textbf{OnOverlapEnd Function}: This function is called when an actor stops overlapping with the trigger volume. It can be used to implement logic for closing the door or deactivating the interactive elements.
\end{itemize}

\emph{\#pragma once}

class ATriggerVolume;

\emph{\#include "CoreMinimal.h"}

\emph{\#include "GameFramework/Actor.h"}

\emph{\#include "DoorTrigger.generated.h"}

UCLASS()

\textbf{class MYGAME\_API} ADoorTrigger : \textbf{public} AActor

\{

~~~~GENERATED\_BODY()

public:

~~~~ADoorTrigger();

protected:

~~~~\textbf{virtual void} BeginPlay() override;

public:

~~~~\textbf{virtual void} Tick(\textbf{float} DeltaTime) override;

private:

~~~~UPROPERTY(VisibleAnywhere)

~~~~ATriggerVolume* DoorTrigger;

~~~~UFUNCTION()

~~~~void OnBeginOverlap(UPrimitiveComponent* OverlappedComponent, AActor* OtherActor, UPrimitiveComponent* OtherComp, int32 OtherBodyIndex, bool bFromSweep, const FHitResult \& SweepResult);

~~~~UFUNCTION()

~~~~void OnEndOverlap(UPrimitiveComponent* OverlappedComponent, AActor* OtherActor, UPrimitiveComponent* OtherComp, int32 OtherBodyIndex);

\};

\emph{\textbf{Listing 1-5}}

DoorTrigger.h

The following methods and functions are crucial for setting up and managing the ADoorTrigger class's behavior in Unreal Engine:

\begin{itemize}
\item
  \textbf{CreateDefaultSubobject}: This function initializes a new ATriggerVolume component named DoorTrigger, which acts as the trigger area for detecting overlaps. This component is set as the root of the actor, ensuring it is the primary element interacting with the game world.
\item
  \textbf{AddDynamic}: AddDynamic is a function used in Unreal Engine to bind a delegate to an event. It allows you to associate an event (like overlap events) with a specific function that will be called when that event is triggered.
\end{itemize}

\emph{\#include "DoorTrigger.h"}

\#include "Engine/TriggerVolume.h"

\emph{\#include "Components/BoxComponent.h"}

ADoorTrigger::ADoorTrigger()

\{

~~~~PrimaryActorTick.bCanEverTick = true;

~~~~DoorTrigger = CreateDefaultSubobject\textless ATriggerVolume\textgreater(TEXT(\emph{"DoorTrigger"}));

~~~~RootComponent = DoorTrigger;

~~~~DoorTrigger-\textgreater OnActorBeginOverlap.AddDynamic(\textbf{this}, \&ADoorTrigger::OnOverlapBegin);

~~~~DoorTrigger-\textgreater OnActorEndOverlap.AddDynamic(\textbf{this}, \&ADoorTrigger::OnOverlapEnd);

\}

\textbf{void} ADoorTrigger::OnOverlapBegin(AActor* OverlappedActor, AActor* OtherActor)

\{

~~~~\emph{// Implement door opening logic here}

\}

\textbf{void} ADoorTrigger::OnOverlapEnd(AActor* OverlappedActor, AActor* OtherActor)

\{

~~~~\emph{// Implement door closing logic here}

\}

\emph{\textbf{Listing 1-6}}

DoorTrigger.cpp

This section explored the fundamentals of blocking movement mechanics in game development. From understanding basic collision concepts to implementing advanced techniques like interactive obstacles, you've gained insights into how collisions drive realistic interactions within game environments.

In the next section, we'll delve deeper into triggering events, exploring how interactions between game objects can dynamically influence gameplay.

\subsection{Triggering Events}

Triggering events involve mechanisms that respond to specific conditions or actions within a game. These events can range from opening doors to activating traps, adding dynamic elements to gameplay.

\subsubsection{Basic Usage}

Triggering events occur when a player character or object interacts with designated trigger volumes or objects in the game world. These interactions typically initiate changes in the game environment or trigger predefined actions.

\subsubsection{Advanced Techniques}

These techniques focus on adding interactive obstacles and dynamic events that react to player actions. For example, you could implement a wall-run detection system, where the player triggers a wall-running animation and movement sequence upon reaching a specific wall. Additionally, combat hit events can be used to trigger damage effects or visual cues when a punch lands or when a character is hit by an attack. Wall climbing triggers can also be used to initiate specific climbing animations or behaviors when the player character touches certain surfaces.

\subsubsection{Practical Applications}

Triggering events are a cornerstone of interactive game design, adding depth and engagement to gameplay. They allow game developers to create dynamic and responsive environments that react to player actions and decisions. By integrating trigger mechanisms, you can enhance the gaming experience in several ways:

\begin{itemize}
\item
  Creating interactive elements like doors, switches, and traps
\item
  Designing puzzles and challenges that require player interaction to progress
\item
  Implementing dynamic game events based on player decisions or game state changes
\item
  Enhancing immersion by simulating cause-and-effect relationships within the game world
\end{itemize}

\subsubsection{Algorithm Overview}

Understanding how to effectively implement triggers involves a systematic approach:

\begin{enumerate}
\def\labelenumi{\arabic{enumi}.}
\item
  1.
\end{enumerate}

\begin{quote}
\textbf{Event Initialization}: Begin by setting up trigger volumes or objects within the game scene. These components are essential for detecting when a player interacts with specific areas or objects.

~
\end{quote}

\begin{enumerate}
\def\labelenumi{\arabic{enumi}.}
\setcounter{enumi}{1}
\item
  2.
\end{enumerate}

\begin{quote}
\textbf{Event Detection}: Continuously monitor for interactions between players and trigger objects. This involves checking for overlap or collision events that indicate a player has interacted with a trigger.

~
\end{quote}

\begin{enumerate}
\def\labelenumi{\arabic{enumi}.}
\setcounter{enumi}{2}
\item
  3.
\end{enumerate}

\begin{quote}
\textbf{Event Resolution}: Execute predefined actions or changes in response to triggered events. This could include opening doors, changing game states, or other actions that affect gameplay.

~
\end{quote}

\begin{enumerate}
\def\labelenumi{\arabic{enumi}.}
\setcounter{enumi}{3}
\item
  4.
\end{enumerate}

\begin{quote}
\textbf{Sequence Handling}: Manage event sequences and dependencies to ensure coherent gameplay progression. Proper sequencing ensures that events unfold logically and that game state changes are handled smoothly.

~
\end{quote}

\subsubsection{Reviewing the Code}

The ATriggerVolume class defines an area in the game world that triggers events when a player or another actor interacts with it. It handles the initialization of the trigger volume and manages the events that occur when actors overlap with it.

The following is a detailed breakdown of the code and its functionality in Listing 1-7:

\begin{itemize}
\item
  \textbf{TriggerVolume Class}: This class represents a trigger volume in the game world. It is derived from AActor, allowing it to be placed in the level and interact with other actors.
\item
  \textbf{BeginPlay Function}: This function is called when the game starts or when the actor is spawned. It is used for any initialization specific to the trigger volume, ensuring that all necessary setup occurs at the start of the game.
\item
  \textbf{Tick Function}: The Tick function is called every frame. While it does not perform any operations in this implementation, it is available for future updates or continuous checks related to the trigger volume.
\item
  \textbf{TriggerBox Property}: This is a UBoxComponent that defines the physical boundary of the trigger volume. It is used to detect overlaps with other actors.
\item
  \textbf{OnOverlapBegin} \textbf{and OnOverlapEnd Functions}: These are callback functions that handle events when actors begin or end their overlap with the trigger volume. They can be used to trigger specific actions, such as opening a door or activating a trap.
\end{itemize}

\emph{\#pragma once}

\emph{class ATriggerVolume}

\emph{\#include "CoreMinimal.h"}

\emph{\#include "GameFramework/Actor.h"}

\emph{\#include "TriggerVolume.generated.h"}

UCLASS()

\textbf{class MYGAME\_API} ATriggerVolume : \textbf{public} AActor

\{

~~~~GENERATED\_BODY()

public:

~~~~ATriggerVolume();

protected:

~~~~\textbf{virtual void} BeginPlay() override;

public:

~~~~\textbf{virtual void} Tick(\textbf{float} DeltaTime) override;

private:

~~~~UPROPERTY(VisibleAnywhere)

~~~~UBoxComponent* TriggerBox;

~~~~UFUNCTION()

~~~~void OnOverlapBegin(UPrimitiveComponent* OverlappedComponent, AActor* OtherActor, UPrimitiveComponent* OtherComp, int32 OtherBodyIndex, bool bFromSweep, const FHitResult\& SweepResult);

~~~~UFUNCTION()

~~~~void OnOverlapEnd(UPrimitiveComponent* OverlappedComponent, AActor* OtherActor, UPrimitiveComponent* OtherComp, int32 OtherBodyIndex, bool bFromSweep, const FHitResult\& SweepResult);

\};

\emph{\textbf{Listing 1-7}}

TriggerVolume.h

Listing 1-8 highlights the most important parts of this code, including CreateDefaultSubobject, SetCollisionEnabled, OnComponentBeginOverlap, OnComponentEndOverlap, OnOverlapBegin, and OnOverlapEnd.

\begin{itemize}
\item
  \textbf{Constructor Initialization}: The CreateDefaultSubobject function creates a UBoxComponent named TriggerBox. This component acts as the collision boundary for the trigger volume. The PrimaryActorTick.bCanEverTick property is set to true to ensure the Tick function is called every frame.
\item
  \textbf{Collision Settings}: The SetCollisionEnabled method is set to ECollisionEnabled::QueryOnly, which means the component will detect overlaps but not participate in physical collisions. SetCollisionResponseToAllChannels is set to ECollisionResponse::ECR\_Overlap, meaning the trigger box will generate overlap events with all channels.
\item
  \textbf{Event Bindings}: The OnComponentBeginOverlap and OnComponentEndOverlap methods are bound to the OnOverlapBegin and OnOverlapEnd functions, respectively. This binding allows the trigger volume to respond to overlap events by executing the specified logic.
\item
  \textbf{Overlap Handlers}: OnOverlapBegin and OnOverlapEnd are placeholder functions where you can implement specific logic for when an actor enters or exits the trigger volume. This might include actions like opening a door, activating a trap, or other game-specific events.
\end{itemize}

\emph{\#include "TriggerVolume.h"}

\emph{\#include "Components/BoxComponent.h"}

ATriggerVolume::ATriggerVolume()

\{

~~~~PrimaryActorTick.bCanEverTick = true;

~~~~TriggerBox = CreateDefaultSubobject\textless UBoxComponent\textgreater(TEXT(\emph{"TriggerBox"}));

~~~~RootComponent = TriggerBox;

~~~~TriggerBox-\textgreater SetCollisionEnabled(ECollisionEnabled::QueryOnly);

~~~~TriggerBox-\textgreater SetCollisionResponseToAllChannels(ECollisionResponse::ECR\_Overlap);

~~~~TriggerBox-\textgreater OnComponentBeginOverlap.AddDynamic(\textbf{this}, \&ThisClass::OnOverlapBegin);

~~~~TriggerBox-\textgreater OnComponentEndOverlap.AddDynamic(\textbf{this}, \& ThisClass::OnOverlapEnd);

\}

\textbf{void} ATriggerVolume::BeginPlay()

\{

~~~~Super::BeginPlay();

\}

\textbf{void} ATriggerVolume::Tick(\textbf{float} DeltaTime)

\{

~~~~Super::Tick(DeltaTime);

\}

\textbf{void} ATriggerVolume::OnOverlapBegin(AActor* OverlappedActor, AActor* OtherActor)

\{

~~~~\emph{// Implement event logic here (e.g., door opening, trap activation)}

\}

\textbf{void} ATriggerVolume::OnOverlapEnd(AActor* OverlappedActor, AActor* OtherActor)

\{

~~~~\emph{// Implement event reset logic here (if needed)}

\}

\emph{\textbf{Listing 1-8}}

TriggerVolume.cpp

\subsubsection{Handling Interactive Elements}

Interactive elements in a game provide a way for players to engage with various objects and mechanisms. These objects can include switches, levers, or any elements that respond to player actions. Here's a detailed look at the AInteractiveObject class and its implementation.

The AInteractiveObject class is designed to represent interactive elements in the game world. Here's a detailed breakdown of its components and functionality:

\begin{itemize}
\item
  \textbf{InteractiveObject Class}: This class inherits from AActor and represents an object that players can interact with. It forms the basis for objects like switches and levers that can be activated or deactivated.
\item
  \textbf{ObjectMesh Property}: This property is a UStaticMeshComponent that provides the visual representation of the interactive object. It is the root component, meaning it forms the base of the actor in the game world.
\item
  \textbf{bIsActive Property}: This boolean flag indicates whether the object is currently active. It helps manage the state of the object and determines whether it should perform specific actions.
\item
  \textbf{Interact Function}: This function handles interactions with the object. It toggles the object's state between active and inactive and executes the corresponding logic based on the current state.
\end{itemize}

\emph{\#pragma once}

class UStaticMeshComponent;

\emph{\#include "CoreMinimal.h"}

\emph{\#include "GameFramework/Actor.h"}

\emph{\#include "InteractiveObject.generated.h"}

UCLASS()

\textbf{class MYGAME\_API} AInteractiveObject : \textbf{public} AActor

\{

~~~~GENERATED\_BODY()

public:

~~~~AInteractiveObject();

protected:

~~~~\textbf{virtual void} BeginPlay() override;

public:

~~~~\textbf{virtual void} Tick(\textbf{float} DeltaTime) override;

private:

~~~~UPROPERTY(VisibleAnywhere, EditDefaultsOnly, Category = "Interactive Object")

~~~~UStaticMeshComponent* ObjectMesh;

~~~~\textbf{bool} bIsActive;

~~~~UFUNCTION()

~~~~\textbf{void} Interact();

\};

\emph{\textbf{Listing 1-9}}

InteractiveObject.h

The provided code outlines the implementation of interactive objects in the game. Here's a closer look at the key components:

\begin{itemize}
\item
  \textbf{Constructor Initialization}: The CreateDefaultSubobject function creates the visual component of the interactive object, and PrimaryActorTick.bCanEverTick ensures that the Tick function is called every frame.
\item
  \textbf{State Management}: The bIsActive flag tracks the object's state, determining whether it is currently active or not. This flag is essential for toggling the object's behavior.
\item
  \textbf{Interaction Logic}: The Interact function defines how the object responds to player interactions. It toggles the object's state and executes specific logic based on whether the object is being activated or deactivated.
\end{itemize}

\emph{\#include "InteractiveObject.h"}

\#include "Components/StaticMeshComponent.h"

AInteractiveObject::AInteractiveObject()

\{

~~~~PrimaryActorTick.bCanEverTick = true;

~~~~ObjectMesh = CreateDefaultSubobject\textless UStaticMeshComponent\textgreater(TEXT("ObjectMesh"));

~~~~RootComponent = ObjectMesh;

~~~~bIsActive = false;

\}

\textbf{void} AInteractiveObject::BeginPlay()

\{

~~~~Super::BeginPlay();

\}

\textbf{void} AInteractiveObject::Tick(\textbf{float} DeltaTime)

\{

~~~~Super::Tick(DeltaTime);

\}

\textbf{void} AInteractiveObject::Interact()

\{

~~~~// If the object is already active, deactivate it and exit the function

~~~~if (bIsActive)

~~~~\{

~~~~~~~~bIsActive = false;~~// Deactivate object (e.g., close door, disable trap)

~~~~~~~~return;~~// Exit as no further actions are needed

~~~~\}

~~~~// If the object is not active, activate it

~~~~bIsActive = true;~~// Activate object (e.g., open door, trigger trap)

~~~~// Implement activation logic here

\}

\emph{\textbf{Listing 1-10}}

InteractiveObject.cpp

\subsubsection{Implementing Dynamic Events}

Unreal Engine's ability to manage dynamic events enables the creation of complex and engaging gameplay scenarios. By using event sequences, you can orchestrate actions and responses that evolve based on specific triggers and conditions, enhancing the interactivity and immersion of the game. The UEventChain class provides a robust mechanism for managing such dynamic events.

\textbf{Advanced Event Sequences}

The UEventChain class facilitates the creation and management of sequences of events that depend on specific conditions or triggers. This class is essential for orchestrating complex event chains, allowing for advanced gameplay scenarios.

The UEventChain class is designed to manage sequences of events, each of which can be triggered with a specific delay. Here's a detailed breakdown of its components:

\begin{itemize}
\item
  \textbf{EventChain Class}: This class extends UObject and manages a sequence of events. It allows for the orchestration of complex event chains by using timers and delegates.
\item
  \textbf{StartEventChain Function}: This function initiates the event sequence, starting with the first event in the EventSequence array. It ensures that the sequence begins when called.
\item
  \textbf{AddEvent Function}: This function adds a new event to the sequence. Each event is associated with a time delay and a delegate function, which is executed when the event is triggered.
\item
  \textbf{EventTimerHandle}: A FTimerHandle is used to manage and track the timing of events. It ensures that events are executed at the appropriate times.
\item
  \textbf{EventSequence Array}: This array holds the delegates for each event in the sequence. It allows for the dynamic scheduling and execution of events.
\item
  \textbf{ExecuteEvent Function}: This function executes the current event and schedules the next one based on the specified delay. It uses the timer manager to handle the timing and sequencing of events.
\end{itemize}

\emph{\#pragma once}

\emph{\#include "CoreMinimal.h"}

\emph{\#include "UObject/NoExportTypes.h"}

\emph{\#include "EventChain.generated.h"}

\emph{/**}

~\emph{*}

~\emph{*/}

UCLASS()

\textbf{class MYGAME\_API} UEventChain : \textbf{public} UObject

\{

~~~~GENERATED\_BODY()

public:

~~~~UEventChain();

~~~~\textbf{void} StartEventChain();

~~~~\textbf{void} AddEvent(\textbf{float} TimeDelay, FTimerDelegate TimerDelegate);

private:

~~~~FTimerHandle EventTimerHandle;

~~~~TArray\textless TTuple\textless float, FTimerDelegate\textgreater\textgreater{} EventSequence;

~~~~\textbf{void} ExecuteEvent(int32 EventIndex);

\};

\emph{\textbf{Listing 1-11}}

EventChain.h

The code implementation for the UEventChain class showcases the management of event sequences with delays. Here's a breakdown of how the code works:

\begin{itemize}
\item
  \textbf{Constructor Initialization}: The default constructor initializes the UEventChain instance, but no specific initialization is required in this case.
\item
  \textbf{ExecuteEvent Function}: This function executes the event at the given index and schedules the next event based on a delay. The lambda function is used to capture the current event index and ensure that events are executed in sequence.
\end{itemize}

\emph{\#include "EventChain.h"}

\emph{\#include "Engine/World.h"}

UEventChain::UEventChain()

\{

\}

\textbf{void} UEventChain::StartEventChain()

\{

~~~~\textbf{if} (EventSequence.Num() \textgreater{} 0)

~~~~\{

~~~~~~~~ExecuteEvent(0);

~~~~\}

\}

void UEventChain::AddEvent(float TimeDelay, FTimerDelegate TimerDelegate)

\{

~~~~// Adding a tuple with TimeDelay and TimerDelegate to the EventSequence

~~~~EventSequence.Add(\{ TimeDelay, TimerDelegate \});

\}

void UEventChain::ExecuteEvent(int32 EventIndex)

\{

~~~~if (EventIndex \textless{} EventSequence.Num())

~~~~\{

~~~~~~~~// Execute the event delegate

~~~~~~~~EventSequence{[}EventIndex{]}.Value.ExecuteIfBound();

~~~~~~~~// Increment event index

~~~~~~~~EventIndex++;

~~~~~~~~// If there are more events, set a timer for the next one

~~~~~~~~if (EventIndex \textless{} EventSequence.Num())

~~~~~~~~\{

~~~~~~~~~~~~float NextDelay = EventSequence{[}EventIndex{]}.Key;

~~~~~~~~~~~~GetWorld()-\textgreater GetTimerManager().SetTimer(EventTimerHandle, {[}this, EventIndex{]}()

~~~~~~~~~~~~\{

~~~~~~~~~~~~~~~~ExecuteEvent(EventIndex);

~~~~~~~~~~~~\}, NextDelay, false);

~~~~~~~~\}

~~~~\}

\}

\emph{\textbf{Listing 1-12}}

EventChain.cpp

This section explored the mechanics of triggering events in game development, from basic event detection to advanced event sequences. You've gained insights into how these mechanisms create interactive and dynamic gameplay experiences within game environments.

In the next section, we will delve into interaction detection. This topic will cover how to effectively detect and manage player interactions with various game elements.

\textbf{Interaction Detection}

Interaction detection in game development refers to identifying when and how players interact with objects within the game world. This mechanism enables dynamic responses to player actions, such as picking up items, opening doors, or triggering conversations.

Imagine your character approaches a treasure chest. Interaction detection allows the game to recognize this proximity and trigger an event where the chest opens, revealing its contents.

\textbf{Basic Usage}

Interaction detection occurs when a player performs an action, like pressing a button, near an interactive object. The system checks if the player is within a specified range or meets certain conditions to trigger the interaction.

This basic application helps in defining how the game responds to player actions, ensuring interactive objects react appropriately.

\textbf{Advanced Techniques}

Advanced techniques involve complex interaction systems where multiple conditions must be met for interactions to occur. These systems can include context-sensitive interactions, where the player's action varies based on the situation or object.

For example, a single button press might open a door if the player is nearby or pick up an item if the player is looking at it.

\textbf{Practical Applications}

Interaction detection is crucial for the following:

\begin{itemize}
\item
  Enabling player-object interactions
\item
  Facilitating dialogue systems and NPC interactions
\item
  Managing inventory systems and item pickups
\item
  Enhancing environmental storytelling through interactive elements
\end{itemize}

\textbf{Algorithm Overview}

Let's visualize the algorithm used for interaction detection:

\begin{enumerate}
\def\labelenumi{\arabic{enumi}.}
\item
  1.
\end{enumerate}

\begin{quote}
\textbf{Initialization}: Set up interactive objects and define their properties within the game scene.

~
\end{quote}

\begin{enumerate}
\def\labelenumi{\arabic{enumi}.}
\setcounter{enumi}{1}
\item
  2.
\end{enumerate}

\begin{quote}
\textbf{Condition Monitoring}: Continuously monitor player actions and proximity to interactive objects.

~
\end{quote}

\begin{enumerate}
\def\labelenumi{\arabic{enumi}.}
\setcounter{enumi}{2}
\item
  3.
\end{enumerate}

\begin{quote}
\textbf{Interaction Detection}: Check if interaction conditions are met (e.g., player is close and presses a button).

~
\end{quote}

\begin{enumerate}
\def\labelenumi{\arabic{enumi}.}
\setcounter{enumi}{3}
\item
  4.
\end{enumerate}

\begin{quote}
\textbf{Event Activation}: Execute predefined actions or changes in response to successful interaction detection.

~
\end{quote}

This algorithm forms the basis for implementing interaction detection mechanics in Unreal Engine.

\textbf{Reviewing the Code}

The InteractiveObject class in Unreal Engine defines an object that players can interact with in the game world. This includes managing initialization, condition checking, and handling interactions. The following sections break down the InteractiveObject class, providing a comprehensive look at its components and functions.

This header file defines the AInteractiveObject class, which inherits from AActor, making it a part of the game world and allowing it to be interacted with.

\begin{itemize}
\item
  \textbf{InteractiveObject Class}: This class represents an object that can be interacted with within the game. It inherits from AActor, meaning it is a game entity with physical presence and behavior.
\item
  \textbf{AInteractiveObject Constructor}: The constructor sets default values and initializes the object's components. It prepares the object to be used in the game by creating a static mesh component and setting the initial state of interaction.
\item
  \textbf{OnInteract Function}: This function is intended to handle interactions with the object. The implementation of the interaction logic is placed here, and it will be executed when the player interacts with the object. It checks if the object is interactable before proceeding.
\end{itemize}

class UStaticMeshComponent;

\emph{\#include "CoreMinimal.h"}

\emph{\#include "GameFramework/Actor.h"}

\emph{\#include "InteractiveObject.generated.h"}

UCLASS()

\textbf{class MYGAME\_API} AInteractiveObject : \textbf{public} AActor

\{

~~~~GENERATED\_BODY()

public:

~~~~AInteractiveObject();

protected:

~~~~\textbf{virtual void} BeginPlay() override;

public:

~~~~\textbf{virtual void} Tick(\textbf{float} DeltaTime) override;

private:

~~~~UPROPERTY(VisibleAnywhere)

~~~~UStaticMeshComponent* ObjectMesh;

~~~~UFUNCTION()

~~~~\textbf{void} OnInteract();

~~~~\textbf{bool} bIsInteractable;

\};

\emph{\textbf{Listing 1-13}}

InteractiveObject.h

Listing 1-14 provides the functionality for the AInteractiveObject class, detailing how it behaves during gameplay. Here are the main components:

\begin{itemize}
\item
  \textbf{CreateDefaultSubobject}: This function creates an instance of UStaticMeshComponent, which represents the visual component of the interactive object. This component is set as the root, ensuring that it is the primary component for rendering and handling collisions.
\item
  \textbf{PrimaryActorTick.bCanEverTick}: This line enables the object to receive updates every frame, allowing it to perform tasks or checks continuously during gameplay.
\item
  \textbf{ObjectMesh Initialization}: The ObjectMesh is assigned to the root component, ensuring that it is the main part of the actor and will be visible in the game world.
\item
  \textbf{bIsInteractable Initialization}: This boolean flag is set to true by default, indicating that the object is interactable. It allows for dynamic changes to the object's interaction state.
\item
  \textbf{Interact Function}: This function defines what happens when the player interacts with the object. It checks if the object is interactable before executing the interaction logic. The specific logic for interactions, such as opening a door or triggering a trap, would be implemented here.
\end{itemize}

\emph{\#include "InteractiveObject.h"}

\emph{\#include "Components/StaticMeshComponent.h"}

AInteractiveObject::AInteractiveObject()

\{

~~~~PrimaryActorTick.bCanEverTick = true;

~~~~ObjectMesh = CreateDefaultSubobject\textless UStaticMeshComponent\textgreater(TEXT(\emph{"ObjectMesh"}));

~~~~RootComponent = ObjectMesh;

~~~~bIsInteractable = true;

\}

\textbf{void} AInteractiveObject::OnInteract()

\{

~~~~\textbf{if} (bIsInteractable)

~~~~\{

~~~~~~~~\emph{// Implement interaction logic here}

~~~~\}

\}

\emph{\textbf{Listing 1-14}}

InteractiveObject.cpp

\textbf{Handling Player Interactions}

The PlayerCharacter class is designed to manage player interactions with interactive objects in the game. It includes setting up input bindings, checking for interactable objects, and handling player input for interactions.

In Listing 1-15, the header file outlines the APlayerCharacter class, which extends the ACharacter class. It is responsible for handling player actions, such as interacting with objects in the game world. It consists of the following components:

\begin{itemize}
\item
  \textbf{PlayerCharacter Class}: This class represents the player's character in the game. It manages player-specific actions, including interacting with objects and handling input.
\item
  \textbf{APlayerCharacter Constructor}: The constructor initializes the player character, enabling it to receive updates every frame by setting PrimaryActorTick.bCanEverTick to true.
\item
  \textbf{BeginPlay Function}: This function is called when the game begins or when the player character is spawned. It is used to set up input bindings. Specifically, it binds the ``Interact'' action to the OnInteract function, allowing the player to perform interactions using a specified input (e.g., a key press).
\item
  \textbf{Tick Function}: This function is called every frame, enabling the character to continuously check for interactable objects. It also ensures that the CheckForInteractableObjects function is executed in each frame to detect nearby interactive elements.
\item
  \textbf{OnInteract Function}: This function is intended to handle the player's interaction input. When the player triggers the interaction action (e.g., by pressing a key), this function will be called to execute the interaction logic.
\item
  \textbf{CheckForInteractableObjects Function}: This function is designed to check for interactive objects in the vicinity of the player. It will be used to determine which objects the player can interact with, potentially involving collision detection or distance checks.
\end{itemize}

\emph{\#pragma once}

\emph{\#include "CoreMinimal.h"}

\emph{\#include "GameFramework/Character.h"}

\emph{\#include "PlayerCharacter.generated.h"}

UCLASS()

\textbf{class MYGAME\_API} APlayerCharacter : \textbf{public} ACharacter

\{

~~~~GENERATED\_BODY()

public:

~~~~APlayerCharacter();

protected:

~~~~\textbf{virtual void} BeginPlay() override;

~~~~\textbf{virtual void} SetupPlayerInputComponent(class UInputComponent* PlayerInputComponent) override;

public:

~~~~\textbf{virtual void} Tick(\textbf{float} DeltaTime) override;

private:

~~~~UFUNCTION()

~~~~\textbf{void} OnInteract();

~~~~\textbf{void} CheckForInteractableObjects();

\};

\emph{\textbf{Listing 1-15}}

PlayerCharacter.h (APlayerCharacter Class)

This implementation file provides the actual functionality for the APlayerCharacter class, detailing how it handles player interactions and checks for nearby interactive objects:

\begin{itemize}
\item
  \textbf{PrimaryActorTick.bCanEverTick}: This line ensures that the player character is updated every frame, allowing it to perform actions and checks continuously.
\item
  \textbf{Timer Setup}: SetTimer(InteractionTimerHandle, this, \&APlayerCharacter::CheckForInteractableObjects, 1.0f, true) will call CheckForInteractableObjects every one second. The true parameter makes it a repeating timer.
\item
  \textbf{OnInteract Function}: This function is a placeholder for the interaction logic. When the player initiates an interaction, this function will be called to execute any relevant actions (e.g., opening a door, picking up an item).
\item
  \textbf{CheckForInteractableObjects Function}: This function is meant to check for nearby interactive objects. The actual implementation will involve logic to detect objects within a certain range or through collision checks, allowing the player to interact with them if they are close enough.
\end{itemize}

\emph{\#include "PlayerCharacter.h"}

\emph{\#include "InteractiveObject.h"}

\emph{\#include "GameFramework/PlayerController.h"}

\#include "Components/InputComponent.h"

APlayerCharacter::APlayerCharacter()

\{

~~~~PrimaryActorTick.bCanEverTick = true;

\}

\textbf{void} APlayerCharacter::BeginPlay()

\{

~~~~Super::BeginPlay();

~~~~InputComponent-\textgreater BindAction(\emph{"Interact"}, IE\_Pressed, \textbf{this}, \&APlayerCharacter::OnInteract);

\}

\textbf{void} APlayerCharacter::Tick(\textbf{float} DeltaTime)

\{

~~~~Super::Tick(DeltaTime);

~~~~CheckForInteractableObjects();

\}

\textbf{void} APlayerCharacter::OnInteract()

\{

~~~~\emph{// Implement interaction logic here}

\}

\textbf{void} APlayerCharacter::CheckForInteractableObjects()

\{

~~~~\emph{// Implement logic to check for nearby interactive objects}

\}

\textbf{void} APlayerCharacter::SetupPlayerInputComponent(UInputComponent* PlayerInputComponent)

\{

~~~~Super::SetupPlayerInputComponent(PlayerInputComponent);

~~~~// Bind the "Interact" action to the OnInteract method

~~~~PlayerInputComponent-\textgreater BindAction("Interact", IE\_Pressed, this, \&APlayerCharacter::OnInteract);

\}

\emph{\textbf{Listing 1-16}}

PlayerCharacter.cpp (APlayerCharacter Class)

\textbf{Implementing Interaction Responses}

In Unreal Engine, interactions can trigger a range of responses, from animations and UI updates to modifications in game state. This section will explore advanced systems for managing complex interactions that respond to multiple conditions and context-sensitive scenarios.

\textbf{Advanced Interaction Systems}

The ComplexInteractionSystem class is designed to manage sophisticated interaction scenarios, handling interactions that are dependent on various conditions and contexts. This class provides a way to set up and manage interactions within the game.

This header file defines the UComplexInteractionSystem class, which extends UObject. It is used for managing complex interactions within the game world.

\begin{itemize}
\item
  \textbf{ComplexInteractionSystem Class}: This class is responsible for managing complex interaction systems within the game. It keeps track of interactive objects and their interaction states, facilitating context-sensitive interactions.
\item
  \textbf{UComplexInteractionSystem Constructor}: The constructor initializes the interaction system. It sets up the internal data structures to manage interactive objects and their states.
\item
  \textbf{AddInteractiveObject Function}: This method allows you to add interactive objects to the interaction system. By maintaining a list of these objects (InteractiveObjects), the system can manage and process interactions for each one.
\item
  \textbf{HandleInteraction Function}: This function is designed to handle interactions based on the type of interaction specified. It uses a map (InteractionStates) to manage and check the states associated with different interaction types.
\end{itemize}

\textbf{Use Cases}

\begin{itemize}
\item
  \textbf{Opening a Door}: When the player interacts with a door (triggered by proximity), the interaction could trigger an animation to open the door and play a sound effect.
\item
  \textbf{Picking Up an Item}: When the player interacts with an item (e.g., a key or health potion), the item is added to the player's inventory and a message is displayed in the UI to notify the player of the pickup.
\item
  \textbf{Starting a Quest}: When the player interacts with a non-playable character (NPC), the interaction could trigger the beginning of a quest, updating the quest log and displaying the relevant dialogue options.
\end{itemize}

\textbf{InteractionType Demonstration}

\begin{itemize}
\item
  \textbf{InteractionType::Activate}: Used when interacting with objects that are meant to be activated, such as buttons, levers, or doors. For example, InteractionType::Activate could trigger the opening of a door or the activation of a trap.
\item
  \textbf{InteractionType::PickUp}: Used for items the player can collect or add to their inventory. For example, InteractionType::PickUp could be used when the player collects a health potion or a weapon from the environment.
\item
  \textbf{InteractionType::Talk}: Used for dialogues with NPCs. For example, InteractionType::Talk would trigger a dialogue sequence with an NPC, possibly triggering a quest or revealing story elements.
\end{itemize}

\emph{\#pragma once}

\emph{\#include "CoreMinimal.h"}

\emph{\#include "UObject/NoExportTypes.h"}

\emph{\#include "ComplexInteractionSystem.generated.h"}

UCLASS()

\textbf{class MYGAME\_API} UComplexInteractionSystem : \textbf{public} UObject

\{

~~~~GENERATED\_BODY()

public:

~~~~UComplexInteractionSystem();

~~~~\textbf{void} AddInteractiveObject(AInteractiveObject* InteractiveObject);

~~~~\textbf{void} HandleInteraction(FString InteractionType);

private:

~~~~TArray\textless AInteractiveObject*\textgreater{} InteractiveObjects;

~~~~TMap\textless FString, \textbf{bool}\textgreater{} InteractionStates;

\};

\emph{\textbf{Listing 1-17}}

ComplexInteractionSystem.h

This implementation file provides the functional details for the UComplexInteractionSystem class, defining how it manages interactions and interactive objects:

\begin{itemize}
\item
  \textbf{UComplexInteractionSystem Constructor}: The constructor ensures that the interaction system is properly initialized. At this stage, no specific functionality is defined beyond basic setup.
\item
  \textbf{AddInteractiveObject Function}: This method adds an AInteractiveObject to the InteractiveObjects array. This allows the interaction system to keep track of all interactive objects in the game, enabling it to process interactions related to these objects.
\end{itemize}

\emph{\#include "ComplexInteractionSystem.h"}

\emph{\#include "InteractiveObject.h"}

UComplexInteractionSystem::UComplexInteractionSystem()

\{

\}

\textbf{void} UComplexInteractionSystem::AddInteractiveObject(AInteractiveObject* InteractiveObject)

\{

~~~~InteractiveObjects.Add(InteractiveObject);

\}

\textbf{void} UComplexInteractionSystem::HandleInteraction(FString InteractionType)

\{

~~~~\emph{// Implement interaction handling logic here}

\}

\emph{\textbf{Listing 1-18}}

ComplexInteractionSystem.cpp

This section explored the mechanics of interaction detection in game development. From understanding basic interaction concepts to implementing advanced techniques like complex interaction systems, you've gained insights into how interactions create engaging and dynamic gameplay experiences.

In the next section, we will delve into physics reactions. Here, we'll examine how physical interactions and reactions can be simulated within your game, adding realism and depth to gameplay through the application of physical forces and responses.

\textbf{Physics Reactions}

Physics reactions in game development refer to how objects respond to physical interactions and forces within the game world. These mechanics ensure realistic behavior when objects collide, fall, or interact with each other, adhering to the laws of physics.

Imagine a game where a character throws a rock at a stack of crates. Physics reactions handle how the rock impacts the crates, causing them to topple over realistically.

\textbf{Basic Usage}

Basic physics reactions involve setting up objects with physical properties, such as mass, friction, and restitution. These properties determine how objects move and interact when forces are applied. This basic application allows developers to simulate realistic behaviors for game objects, enhancing immersion.

\textbf{Advanced Techniques}

Advanced techniques include simulating complex interactions like ragdoll physics, where characters' bodies react dynamically to collisions, or soft-body physics, where objects deform upon impact.

For example, when a character is knocked down by an explosion, ragdoll physics make the fall look realistic as limbs react naturally to forces.

\textbf{Practical Applications}

Physics reactions are crucial for

\begin{itemize}
\item
  Simulating realistic object behavior and interactions
\item
  Enhancing gameplay with dynamic environmental reactions
\item
  Creating believable character movements and responses
\item
  Implementing destructible environments and objects
\end{itemize}

\textbf{Algorithm Overview}

Let's visualize the algorithm used for physics reactions:

\begin{enumerate}
\def\labelenumi{\arabic{enumi}.}
\item
  1.
\end{enumerate}

\begin{quote}
\textbf{Initialization}: Set up objects with physical properties within the game scene.

~
\end{quote}

\begin{enumerate}
\def\labelenumi{\arabic{enumi}.}
\setcounter{enumi}{1}
\item
  2.
\end{enumerate}

\begin{quote}
\textbf{Force Application}: Apply forces to objects based on player actions or environmental factors.

~
\end{quote}

\begin{enumerate}
\def\labelenumi{\arabic{enumi}.}
\setcounter{enumi}{2}
\item
  3.
\end{enumerate}

\begin{quote}
\textbf{Collision Detection}: Continuously check for collisions between objects.

~
\end{quote}

\begin{enumerate}
\def\labelenumi{\arabic{enumi}.}
\setcounter{enumi}{3}
\item
  4.
\end{enumerate}

\begin{quote}
\textbf{Physics Calculation}: Calculate the resulting motion and reactions based on physical properties and applied forces.

~
\end{quote}

\begin{enumerate}
\def\labelenumi{\arabic{enumi}.}
\setcounter{enumi}{4}
\item
  5.
\end{enumerate}

\begin{quote}
\textbf{Visual Update}: Update the visual representation of objects to reflect their physical state.

~
\end{quote}

This algorithm forms the basis for implementing physics reaction mechanics in Unreal Engine.

\textbf{Reviewing the Code}

The PhysicalObject class demonstrates how to define and manage physical objects within a game environment. This involves handling initialization, applying forces, and managing collision responses.

This class represents an object with physical properties and behavior, including physics simulation and collision handling.

\begin{itemize}
\item
  \textbf{APhysicalObject Class}: Defines an object with physical properties. It manages initialization and setup related to physics simulation.
\item
  \textbf{BeginPlay Function}: Called when the game starts, initializing the physical object.
\item
  \textbf{Tick Function}: Updates the object's state every frame.
\item
  \textbf{ObjectMesh}: A UStaticMeshComponent representing the visual and physical representation of the object.
\item
  \textbf{PhysicsConstraint}: A component used to define and manage physics constraints for the object.
\end{itemize}

\emph{\#pragma once}

class UStaticMeshComponent;

\emph{\#include "CoreMinimal.h"}

\emph{\#include "GameFramework/Actor.h"}

\emph{\#include "PhysicalObject.generated.h"}

UCLASS()

\textbf{class MYGAME\_API} APhysicalObject : \textbf{public} AActor

\{

~~~~GENERATED\_BODY()

public:

~~~~APhysicalObject();

private:

~~~~UPROPERTY(VisibleAnywhere)

~~~~UStaticMeshComponent* ObjectMesh;

~~~~UPROPERTY(VisibleAnywhere)

~~~~UPhysicsConstraintComponent* PhysicsConstraint;

\};

\emph{\textbf{Listing 1-19}}

PhysicalObject.h

This setup ensures that the PhysicalObject is fully prepared for interaction with the game world, capable of responding to forces and collisions effectively.

\begin{itemize}
\item
  \textbf{APhysicalObject Constructor}: Initializes the object, setting up physics simulation and collision detection
\item
  \textbf{SetSimulatePhysics}: Enables physics simulation for the object mesh, allowing it to react to physical forces
\item
  \textbf{SetCollisionEnabled}: Enables collision detection and physics responses
\end{itemize}

\emph{\#include "PhysicalObject.h"}

\emph{\#include "Components/StaticMeshComponent.h"}

\emph{\#include "PhysicsEngine/PhysicsConstraintComponent.h"}

APhysicalObject::APhysicalObject()

\{

~~~~PrimaryActorTick.bCanEverTick = true;

~~~~ObjectMesh = CreateDefaultSubobject\textless UStaticMeshComponent\textgreater(TEXT(\emph{"ObjectMesh"}));

~~~~RootComponent = ObjectMesh;

~~~~PhysicsConstraint = CreateDefaultSubobject\textless UPhysicsConstraintComponent\textgreater(TEXT(\emph{"PhysicsConstraint"}));

~~~~PhysicsConstraint-\textgreater SetupAttachment(RootComponent);

~~~~ObjectMesh-\textgreater SetSimulatePhysics(true);

~~~~ObjectMesh-\textgreater SetCollisionEnabled(ECollisionEnabled::QueryAndPhysics);

\}

\textbf{void} APhysicalObject::BeginPlay()

\{

~~~~Super::BeginPlay();

\}

\textbf{void} APhysicalObject::Tick(\textbf{float} DeltaTime)

\{

~~~~Super::Tick(DeltaTime);

\}

\emph{\textbf{Listing 1-20}}

PhysicalObject.cpp (PhysicalObject Class)

\textbf{Applying Forces}

The ApplyForce function allows you to dynamically influence physical objects in the game world by applying forces based on interactions or environmental changes. This function checks if the object is simulating physics before applying the force, ensuring that only appropriate objects respond to such interactions.

Additionally, you can think of the force's direction as being dynamic---often, a force can be applied along the object's forward vector (i.e., its direction) or along its normal. This gives you flexibility in defining how forces affect the object, such as pushing it forward or applying force in response to external stimuli.

This function is crucial for creating responsive and interactive environments, where objects react to player actions or environmental stimuli in a realistic manner.

\begin{itemize}
\item
  \textbf{ApplyForce Function}: Applies a specified force to the object if it is simulating physics
\end{itemize}

void APhysicalObject::ApplyForce(FVector Force)

\{

~~~~// Ensure that physics simulation is enabled

~~~~if (ObjectMesh \&\& ObjectMesh-\textgreater IsSimulatingPhysics())

~~~~\{

~~~~~~~~// Optionally, adjust force based on direction (forward, normal, etc.)

~~~~~~~~// Example of applying force based on forward vector:

~~~~~~~~FVector ForceDirection = ObjectMesh-\textgreater GetForwardVector();

~~~~~~~// Apply force in the chosen direction, you can scale it as needed

~~~~~~~~ObjectMesh-\textgreater AddForce(ForceDirection * Force.Size()); // Apply scaled force in the direction of the forward vector

~~~~\}

\}

\emph{\textbf{Listing 1-21}}

PhysicalObject.cpp (ApplyForce Function)

\textbf{Handling Collisions}

The NotifyHit function is designed to manage the behavior of objects when they collide with other objects in the game world. By implementing custom logic within this function, you can tailor the reactions to collisions, such as triggering effects or changing the object's state.

This function enables you to define how your objects interact upon impact, contributing to a more immersive and dynamic gameplay experience.

\begin{itemize}
\item
  \textbf{NotifyHit Function}: Called when the object collides with another object, allowing for custom collision reaction logic
\end{itemize}

void APhysicalObject::NotifyHit(

~~~~class UPrimitiveComponent* MyComp,

~~~~AActor* Other,

~~~~class UPrimitiveComponent* OtherComp,

~~~~bool bSelfMoved,

~~~~FVector HitLocation,

~~~~FVector HitNormal,

~~~~FVector NormalImpulse,

~~~~const FHitResult\& Hit)

\{

~~~~// Check if the object is in a state to react to collisions

~~~~if (bCanReactToCollision)

~~~~\{

~~~~~~~~// Trigger collision reactions based on custom logic

~~~~~~~~// For example, react to specific objects or types of collisions

~~~~~~~~if (Other \&\& Other-\textgreater IsA(APlayerCharacter::StaticClass()))

~~~~~~~~\{

~~~~~~~~~~~~// Handle player interaction with the object

~~~~~~~~~~~~// Example: Apply damage, trigger event, etc.

~~~~~~~~~~~~HandlePlayerCollision(Other);

~~~~~~~~\}

~~~~~~~~// Optional: Reset the flag after a short delay to prevent multiple reactions

~~~~~~~~// This can be done using a timer or simple flag manipulation

~~~~~~~~bCanReactToCollision = false;

~~~~~~~~// Example cooldown (5 seconds cooldown, customizable)

~~~~~~~~GetWorld()\textgreater GetTimerManager().SetTimer(CollisionCooldownTimerHandle, this, \&APhysicalObject::ResetCollisionFlag, 5.f, false);

~~~~\}

\}

void APhysicalObject::ResetCollisionFlag()

\{

~~~~// Enable collision reactions again after cooldown

~~~~bCanReactToCollision = true;

\}

void APhysicalObject::HandlePlayerCollision(AActor* OtherActor)

\{

~~~~// Implement logic for when the object collides with the player

~~~~// This could be damaging the player, triggering a sound effect, etc.

~~~~if (OtherActor)

~~~~\{

~~~~~~~~// Example: Trigger a specific response, like damage

~~~~~~~~APlayerCharacter* Player = Cast\textless APlayerCharacter\textgreater(OtherActor);

~~~~~~~~if (Player)

~~~~~~~~\{

~~~~~~~~~~~~// Damage logic or other interaction logic

~~~~~~~~~~~~Player-\textgreater ApplyDamage(10);

~~~~~~~~\}

~~~~\}

\}

\emph{\textbf{Listing 1-22}}

PhysicalObject.cpp (NotifyHit Function)

\textbf{Advanced Physics Reactions}

In Unreal Engine, ragdoll physics can be directly controlled within the ACharacter class. By setting the character's skeletal mesh to simulate physics and adjusting the collision profile, characters can dynamically respond to physical impacts. This method enhances realism by allowing the character's body to react naturally to collisions, without needing additional classes or complex setups.

This section explains the components and functionality of the updated player class:

\begin{itemize}
\item
  \textbf{SetCollisionProfileName(``Ragdoll'')}: This sets the collision profile to "Ragdoll", which is typically predefined in Unreal Engine. This profile ensures that the character's mesh behaves like a ragdoll, with appropriate physics and collision handling.
\item
  \textbf{SetAllBodiesBelowSimulatePhysics(``root'', true)}: This enables physics simulation on all bones below the "root" bone in the character's skeleton, simulating full ragdoll behavior.
\item
  \textbf{DisableMovement()}: Disables character movement after the ragdoll is activated to prevent player control while the ragdoll physics take over.
\end{itemize}

void APlayerCharacter::EnableRagdoll()

\{

~~~~// Set the collision profile to \textquotesingle Ragdoll\textquotesingle{} for the mesh

~~~~GetMesh()-\textgreater SetCollisionProfileName("Ragdoll");

~~~~// Enable physics simulation on all bones below the root

~~~~GetMesh()-\textgreater SetAllBodiesBelowSimulatePhysics("root", true);

~~~~// Optionally, disable character movement once ragdoll is activated

~~~~GetCharacterMovement()-\textgreater DisableMovement();

\}

\emph{\textbf{Listing 1-23}}

PlayerCharacter.cpp (Updated)

This section explored the fundamentals of physics reactions in game development. We covered the essential concepts of physical properties, the application of forces, collision handling, and advanced techniques like ragdoll physics. By integrating these mechanics, you can create a more realistic and immersive game experience.

In the next section, we will delve into \textbf{environmental effects}. We will examine how to incorporate dynamic environmental elements to further enhance the gameplay experience and create more engaging and interactive game worlds.

\textbf{Environmental Effects}

Environmental effects in game development refer to the simulation of various natural phenomena and environmental interactions that affect gameplay and enhance the immersive experience. These effects can range from weather conditions and lighting changes to destructible environments and dynamic terrain deformation.

Imagine a game where a character navigates through a forest during a storm. The environmental effects, such as rain, wind, and lightning, create a dynamic and immersive atmosphere.

\textbf{Basic Usage}

Basic environmental effects involve setting up static and dynamic elements that simulate natural phenomena. These elements can include weather conditions like rain and snow, lighting effects like day-night cycles, and interactive environment components like destructible objects.

This basic application allows developers to create realistic and immersive environments that respond to player actions and enhance the overall gameplay experience.

\textbf{Advanced Techniques}

Advanced techniques include the use of particle systems for detailed weather effects, dynamic lighting for real-time changes in the environment, and procedural generation for creating varied and expansive landscapes.

For instance, procedural generation can be applied to create a vast and diverse forest where each playthrough presents a unique exploration experience. Additionally, spawning various visual effects like muzzle flashes, bullet traces, and aura effects can enhance combat and magical encounters, bringing more dynamism and immersion into the game world.

\textbf{Practical Applications}

Environmental effects are crucial for

\begin{itemize}
\item
  Creating immersive and realistic game worlds
\item
  Enhancing gameplay with dynamic and interactive environments
\item
  Simulating natural phenomena and environmental interactions
\item
  Adding depth and complexity to game environments
\end{itemize}

\textbf{Algorithm Overview}

Let's visualize the algorithm used for environmental effects:

\begin{enumerate}
\def\labelenumi{\arabic{enumi}.}
\item
  1.
\end{enumerate}

\begin{quote}
\textbf{Initialization}: Set up the environment with static and dynamic elements.

~
\end{quote}

\begin{enumerate}
\def\labelenumi{\arabic{enumi}.}
\setcounter{enumi}{1}
\item
  2.
\end{enumerate}

\begin{quote}
\textbf{Effect Application}: Apply environmental effects based on predefined conditions or player actions.

~
\end{quote}

\begin{enumerate}
\def\labelenumi{\arabic{enumi}.}
\setcounter{enumi}{2}
\item
  3.
\end{enumerate}

\begin{quote}
\textbf{Interaction Handling}: Implement interactions between the player and environmental elements.

~
\end{quote}

\begin{enumerate}
\def\labelenumi{\arabic{enumi}.}
\setcounter{enumi}{3}
\item
  4.
\end{enumerate}

\begin{quote}
\textbf{Visual Update}: Continuously update the visual representation of environmental effects.

~
\end{quote}

\begin{enumerate}
\def\labelenumi{\arabic{enumi}.}
\setcounter{enumi}{4}
\item
  5.
\end{enumerate}

\begin{quote}
\textbf{Dynamic Changes}: Adjust environmental effects in real time based on game events or player actions.

~
\end{quote}

This algorithm forms the basis for implementing environmental effects in Unreal Engine.

\textbf{Reviewing the Code}

The WeatherEffect class is designed to simulate environmental weather effects like rain and snow. This involves initializing the effect, applying it within the game environment, and managing visual updates through the game loop.

Listing 1-24 explains the following components and functionality of the WeatherEffect class:

\begin{itemize}
\item
  \textbf{WeatherEffect Class}: Represents an object that simulates weather effects using particle systems
\item
  \textbf{UParticleSystemComponent}: Manages the particle system that simulates the weather effect
\end{itemize}

\emph{\#pragma once}

class UParticleSystemComponent;

\emph{\#include "CoreMinimal.h"}

\emph{\#include "GameFramework/Actor.h"}

\emph{\#include "WeatherEffect.generated.h"}

UCLASS()

\textbf{class MYGAME\_API} AWeatherEffect : \textbf{public} AActor

\{

~~~~GENERATED\_BODY()

public:

~~~~AWeatherEffect();

protected:

~~~~\textbf{virtual void} BeginPlay() override;

public:

~~~~\textbf{virtual void} Tick(\textbf{float} DeltaTime) override;

private:

~~~~// Use EditDefaultsOnly if this property is set in the Blueprint and not at runtime

~~~~\textbf{UPROPERTY}(EditDefaultsOnly, BlueprintReadOnly, Category = "Weather Effects")

~~~~\textbf{UParticleSystemComponent}* WeatherParticleSystem;\};

\emph{\textbf{Listing 1-24}}

WeatherEffect.h

This section explains the implementation details in the WeatherEffect.cpp file:

\begin{itemize}
\item
  \textbf{AWeatherEffect Constructor}: Initializes the weather effect, setting up the particle system for simulation
\item
  \textbf{BeginPlay Function}: Sets up initial conditions for the weather effect at the start of the game
\item
  \textbf{Tick Function}: Continuously updates the weather effect, allowing for dynamic changes throughout the game
\end{itemize}

\emph{\#include "WeatherEffect.h"}

\emph{\#include "Particles/ParticleSystemComponent.h"}

\#include "UObject/ConstructorHelpers.h"

AWeatherEffect::AWeatherEffect()

\{

~~~~PrimaryActorTick.bCanEverTick = true;

~~~~// Create a default subobject for the weather particle system

~~~~WeatherParticleSystem = CreateDefaultSubobject\textless UParticleSystemComponent\textgreater(TEXT("WeatherParticleSystem"));

~~~~RootComponent = WeatherParticleSystem;

~~~~// Binding a static particle system asset (hinted for static mesh binding)

~~~~static ConstructorHelpers::FObjectFinder\textless UParticleSystem\textgreater{} ParticleAsset(TEXT("ParticleSystem\textquotesingle/Game/Myfiles/P\_Explosion.P\_Explosion\textquotesingle"));

~~~~if (ParticleAsset.Succeeded())

~~~~\{

~~~~~~~~WeatherParticleSystem-\textgreater SetTemplate(ParticleAsset.Object);~~// setting the particle asset to the component

~~~~\}

\}

\textbf{void} AWeatherEffect::BeginPlay()

\{

~~~~Super::BeginPlay();

\}

\textbf{void} AWeatherEffect::Tick(\textbf{float} DeltaTime)

\{

~~~~Super::Tick(DeltaTime);

\}

\emph{\textbf{Listing 1-25}}

WeatherEffect.cpp (WeatherEffect Class)

\textbf{Applying Environmental Effects}

The EffectApplication function is responsible for applying environmental effects based on predefined conditions or player actions.

This section explains the ApplyWeatherEffect function:

\begin{itemize}
\item
  \textbf{ApplyWeatherEffect Function}: Sets the appropriate particle system template based on the specified weather effect type (e.g., rain or snow)
\end{itemize}

\textbf{void} AWeatherEffect::ApplyWeatherEffect(FString EffectType)

\{

~~~~if (EffectType == "Rain")

~~~~\{

~~~~~~~~// Assigning the rain particle system to the WeatherParticleSystem component

~~~~~~~~static ConstructorHelpers::FObjectFinder\textless UParticleSystem\textgreater{} RainParticleAsset(TEXT("ParticleSystem\textquotesingle/Game/Myfiles/P\_Rain.P\_Rain\textquotesingle"));

~~~~~~~~if (RainParticleAsset.Succeeded())

~~~~~~~~\{

~~~~~~~~~~~~WeatherParticleSystem-\textgreater SetTemplate(RainParticleAsset.Object);

~~~~~~~~\}

~~~~\}

~~~~else if (EffectType == "Snow")

~~~~\{

~~~~~~~~// Assigning the snow particle system to the WeatherParticleSystem component

~~~~~~~~static ConstructorHelpers::FObjectFinder\textless UParticleSystem\textgreater{} SnowParticleAsset(TEXT("ParticleSystem\textquotesingle/Game/Myfiles/P\_Snow.P\_Snow\textquotesingle"));

~~~~~~~~if (SnowParticleAsset.Succeeded())

~~~~~~~~\{

~~~~~~~~~~~~WeatherParticleSystem-\textgreater SetTemplate(SnowParticleAsset.Object);

~~~~~~~~\}

~~~~\}

\}

\emph{\textbf{Listing 1-26}}

WeatherEffect.cpp (ApplyWeatherEffect Function)

\textbf{Handling Interactions}

The OnPlayerEnter function is called when the player enters a specific area, triggering the application of a weather effect (e.g., rain). See Listing 1-27.

\textbf{void} AWeatherEffect::OnPlayerEnter(AActor* PlayerActor)

\{

~~~~\emph{// Implement logic for interacting with the player}

~~~~ApplyWeatherEffect(\emph{"Rain"});

\}

\emph{\textbf{Listing 1-27}}

WeatherEffect.cpp (for OnPlayerEnter Function)

\textbf{Advanced Environmental Effects}

The DynamicLighting class is designed to manage dynamic lighting changes within the game, such as simulating day-night cycles. It includes methods for adjusting lighting in real time based on different conditions.

This section explains the components and functionality of the DynamicLighting header class:

\begin{itemize}
\item
  \textbf{DynamicLighting Class}: Manages dynamic lighting changes in the game environment
\item
  \textbf{UpdateLighting Function}: Adjusts lighting settings based on the time of day, simulating day-night cycles
\end{itemize}

\emph{\#pragma once}

\emph{\#include "CoreMinimal.h"}

\emph{\#include "UObject/NoExportTypes.h"}

\emph{\#include "DynamicLighting.generated.h"}

UCLASS()

\textbf{class MYGAME\_API} UDynamicLighting : \textbf{public} UObject

\{

~~~~GENERATED\_BODY()

public:

~~~~UDynamicLighting();

~~~~\textbf{void} UpdateLighting(\textbf{float} TimeOfDay);

\};

\emph{\textbf{Listing 1-28}}

DynamicLighting.h

This section explains the implementation details in the DynamicLighting.cpp file:

\begin{itemize}
\item
  \textbf{DynamicLighting Class}: Manages dynamic lighting changes in the game environment
\item
  \textbf{UpdateLighting Function}: Adjusts lighting settings based on the time of day, enabling a realistic day-night cycle within the game
\end{itemize}

\emph{\#include "DynamicLighting.h"}

\emph{\#include "Engine/DirectionalLight.h"}

UDynamicLighting::UDynamicLighting()

\{

\}

\textbf{void} UDynamicLighting::UpdateLighting(\textbf{float} TimeOfDay)

\{

~~~~\emph{// Implement dynamic lighting changes based on the time of day}

~~~~\textbf{if} (TimeOfDay \textless{} 12.0f)

~~~~\{

~~~~~~~~\emph{// Morning lighting setup}

~~~~\}

~~~~\textbf{else}

~~~~\{

~~~~~~~~\emph{// Evening lighting setup}

~~~~\}

\}

\emph{\textbf{Listing 1-29}}

DynamicLighting.cpp

This section explored the fundamentals of environmental effects in game development. From understanding basic weather effects to implementing advanced techniques like dynamic lighting and procedural generation, you've gained insights into how environmental mechanics enhance realism and immersion in games.

In the upcoming section, we will delve into hit detection. This topic covers the essential process of determining interactions between objects, which is crucial for responsive gameplay. We will examine both basic and advanced techniques for hit detection, exploring how they contribute to the accuracy and depth of game mechanics.

\textbf{Hit Detection}

Hit detection in game development refers to the process of determining when and where interactions occur between objects, typically involving combat or collisions. Effective hit detection is crucial for ensuring accurate and responsive gameplay, especially in action-oriented games.

Imagine a scenario where a character swings a sword, and the game needs to determine if the sword hits an enemy. Hit detection ensures that these interactions are detected accurately and in real time.

\textbf{Basic Usage}

Basic hit detection involves checking for collisions or overlaps between objects using bounding volumes such as spheres, boxes, or capsules. This technique is straightforward and efficient, making it suitable for many types of games.

This basic application allows developers to implement core gameplay mechanics, such as detecting hits in combat or interactions with the environment.

\textbf{Advanced Techniques}

Advanced hit detection techniques include raycasting, pixel-perfect collision detection, and skeletal hitboxes. These methods provide more precision and can handle complex scenarios, such as detecting hits on specific body parts or in fast-paced action sequences.

For example, skeletal hitboxes can be used to detect hits on different parts of a character's body, allowing for more nuanced damage calculations.

Additionally, you can make use of custom collision channels to handle specialized interactions. By defining your own collision channels, you can fine-tune what objects or characters can interact with each other, providing more control over hit detection and collision resolution. This is especially useful for creating unique gameplay mechanics, such as different damage types or detecting interactions with non-physical objects like triggers or events.

\textbf{Practical Applications}

Hit detection is essential for

\begin{itemize}
\item
  Implementing combat mechanics
\item
  Detecting collisions and interactions
\item
  Calculating damage and effects
\item
  Enhancing gameplay realism and responsiveness
\end{itemize}

\textbf{Algorithm Overview}

Let's visualize the algorithm used for hit detection:

\begin{enumerate}
\def\labelenumi{\arabic{enumi}.}
\item
  1.
\end{enumerate}

\begin{quote}
\textbf{Initialization}: Set up the objects involved in hit detection.

~
\end{quote}

\begin{enumerate}
\def\labelenumi{\arabic{enumi}.}
\setcounter{enumi}{1}
\item
  2.
\end{enumerate}

\begin{quote}
\textbf{Collision Detection}: Continuously check for collisions or overlaps between objects.

~
\end{quote}

\begin{enumerate}
\def\labelenumi{\arabic{enumi}.}
\setcounter{enumi}{2}
\item
  3.
\end{enumerate}

\begin{quote}
\textbf{Hit Processing}: When a hit is detected, apply appropriate responses (e.g., damage, effects).

~
\end{quote}

\begin{enumerate}
\def\labelenumi{\arabic{enumi}.}
\setcounter{enumi}{3}
\item
  4.
\end{enumerate}

\begin{quote}
\textbf{Interaction Handling}: Implement additional logic based on the type of hit (e.g., critical hits, environmental interactions).

~
\end{quote}

\begin{enumerate}
\def\labelenumi{\arabic{enumi}.}
\setcounter{enumi}{4}
\item
  5.
\end{enumerate}

\begin{quote}
\textbf{Visual Update}: Update the visual representation of hits, such as showing damage effects or animations.

~
\end{quote}

This algorithm forms the basis for implementing hit detection in Unreal Engine.

\textbf{Reviewing the Code}

The HitDetection class is designed to manage hit detection logic, including initialization, collision detection, and hit processing. It utilizes a sphere component to detect overlaps and respond accordingly.

\textbf{Setting Up Hit Detection}

The HitDetection class defines an object that manages hit detection within the game. It handles initialization, collision detection, and processing of hit events.

This section explains the components and functionality of the HitDetection class:

\begin{itemize}
\item
  \textbf{HitDetection Class}: Represents an object that manages hit detection using a sphere component
\item
  \textbf{USphereComponent}: Manages the sphere used for detecting overlaps
\item
  \textbf{OnComponentBeginOverlap}: Event triggered when an object overlaps with the sphere component
\end{itemize}

\emph{\#pragma once}

class USphereComponent;

\emph{\#include "CoreMinimal.h"}

\emph{\#include "GameFramework/Actor.h"}

\emph{\#include "HitDetection.generated.h"}

UCLASS()

\textbf{class MYGAME\_API} AHitDetection : \textbf{public} AActor

\{

~~~~GENERATED\_BODY()

public:

~~~~AHitDetection();

protected:

~~~~\textbf{virtual void} BeginPlay() override;

public:

~~~~\textbf{virtual void} Tick(\textbf{float} DeltaTime) override;

private:

~~~~UPROPERTY(VisibleAnywhere)

~~~~USphereComponent* HitSphere;

~~~~UFUNCTION()

~~~~\textbf{void} OnOverlapBegin(\textbf{class UPrimitiveComponent}* OverlappedComp, \textbf{class AActor}* OtherActor, \textbf{class UPrimitiveComponent}* OtherComp, int32 OtherBodyIndex, \textbf{bool} bFromSweep, \textbf{const} FHitResult \& SweepResult);

\};

\emph{\textbf{Listing 1-30}}

HitDetection.h

This section explains the implementation details in the HitDetection.cpp file:

\begin{itemize}
\item
  \textbf{AHitDetection Constructor}: Initializes the hit detection component and sets up the sphere for collision detection
\item
  \textbf{BeginPlay Function}: Sets up initial conditions for hit detection when the game starts
\item
  \textbf{Tick Function}: Continuously updates the hit detection logic every frame
\item
  \textbf{OnOverlapBegin Function}: Handles the logic for processing hits when an overlap event is detected by the sphere component
\end{itemize}

\emph{\#include "HitDetection.h"}

\emph{\#include "Components/SphereComponent.h"}

AHitDetection::AHitDetection()

\{

~~~~PrimaryActorTick.bCanEverTick = true;

~~~~HitSphere = CreateDefaultSubobject\textless USphereComponent\textgreater(TEXT(\emph{"HitSphere"}));

~~~~RootComponent = HitSphere;

~~~~HitSphere-\textgreater OnComponentBeginOverlap.AddDynamic(\textbf{this}, \&AHitDetection::OnOverlapBegin);

\}

\textbf{void} AHitDetection::BeginPlay()

\{

~~~~Super::BeginPlay();

\}

\textbf{void} AHitDetection::Tick(\textbf{float} DeltaTime)

\{

~~~~Super::Tick(DeltaTime);

\}

\textbf{void} AHitDetection::OnOverlapBegin(UPrimitiveComponent* OverlappedComp, AActor* OtherActor, UPrimitiveComponent* OtherComp, int32 OtherBodyIndex, bool bFromSweep, const FHitResult \& SweepResult)

\{

~~~~// If the hit has already been processed, exit the function

~~~~if (bHasHit)

~~~~\{

~~~~~~~~return;

~~~~\}

~~~~// Process the hit

~~~~// Implement hit processing logic here

~~~~// Set the flag to true to prevent reprocessing the same hit

~~~~bHasHit = true;

\}

\emph{\textbf{Listing 1-31}}

HitDetection.cpp

\textbf{Implementing Hit Processing}

The ProcessHit function is responsible for handling the logic of processing hits and applying effects, such as damage or status effects, to the hit actor.

This section explains the ProcessHit function:

\begin{itemize}
\item
  \textbf{ProcessHit Function}: Applies damage or effects to the hit actor when a hit is detected
\end{itemize}

\textbf{void} AHitDetection::ProcessHit(AActor* HitActor)

\{

~~~~\textbf{if} (HitActor)

~~~~\{

~~~~~~~~\emph{// Implement logic for applying damage or effects to the hit actor}

~~~~\}

\}

\emph{\textbf{Listing 1-32}}

HitDetection.cpp (for the ProcessHit Function)

\textbf{Advanced Hit Detection Techniques}

Raycasting is a technique used for precise hit detection, particularly useful in fast-paced action games where accuracy is critical.

This section explains the PerformRaycast function:

\begin{itemize}
\item
  \textbf{PerformRaycast Function}: Performs a raycast from the actor's location forward, detecting any blocking hits along the way.
\item
  \textbf{LineTraceSingleByChannel}: Executes the raycast and checks for collisions along the line, allowing precise hit detection.
\item
  \textbf{Custom Collision Channels}: Replace ECC\_GameTraceChannel1 with the custom collision channel you defined in your project settings. Custom collision channels provide precise control over how raycasts interact with objects. If a trace hits an object with a channel set to Ignore, it bypasses the object without detecting a hit, ideal for background elements or non-interactive objects. With an Overlap response, the trace detects the object but doesn't block or collide with it, perfect for triggers or non-blocking interactions. A Block response ensures the trace stops when it hits an object in the custom channel, useful for barriers or walls. In the Details Panel of an Actor's component, you can configure the collision presets to your custom channel and specify whether it should Block, Overlap, or Ignore for specific channels, allowing you to fine-tune how different objects interact in the game world. For instance, you could set a character's weapons to block certain objects while ignoring others, ensuring dynamic and customized interactions.
\end{itemize}

\textbf{void} AHitDetection::PerformRaycast()

\{

~~~~FVector Start = GetActorLocation();

~~~~FVector ForwardVector = GetActorForwardVector();

~~~~FVector End = ((ForwardVector * 1000.0f) + Start);

~~~~FHitResult HitResult;

~~~~FCollisionQueryParams CollisionParams;

~~~~CollisionParams.AddIgnoredActor(\textbf{this});

~~~~// Use a custom collision channel for more precise hit detection

~~~~// Replace ECC\_MyCustomChannel with your actual custom collision channel

~~~~GetWorld()-\textgreater LineTraceSingleByChannel(HitResult, Start, End, ECC\_MyCustomChannel, CollisionParams);

if (HitResult.bBlockingHit)

\{

~~~~// Process the hit - Implement your logic for interaction with the object

\}

~~~~\textbf{if} (HitResult.bBlockingHit)

~~~~\{

~~~~~~~~\emph{// Implement logic for processing the raycast hit}

~~~~\}

\}

\emph{\textbf{Listing 1-33}}

HitDetection.cpp (for the PerformRaycast Function)

This section explored the fundamentals of hit detection in game development. From understanding basic collision detection to implementing advanced techniques like raycasting, you've gained insights into how hit detection drives accurate and responsive interactions within games.

In the next section, we will explore item collection, a fundamental game mechanic that enhances player engagement through the collection of various items. We will examine how item collection mechanics can be implemented and how they contribute to a rewarding gameplay experience.

\textbf{Item Collection}

Item collection is a fundamental mechanic in many games, enabling players to interact with the game world by picking up items. These items can range from simple collectibles like coins and health packs to more complex objects like weapons and power-ups. Effective item collection mechanics enhance player engagement and reward exploration.

Imagine you're designing a treasure hunt game. Players need to collect items scattered across the game world to progress or gain advantages. Item collection mechanics ensure that these interactions are smooth and rewarding.

\textbf{Basic Usage}

Basic item collection involves detecting when the player character collides with or interacts with an item. Upon collision, the item is collected, and any associated effects (e.g., increasing the player's score or health) are applied.

This basic application allows developers to implement core gameplay mechanics, such as rewarding players for exploration and interaction.

\textbf{Advanced Techniques}

Advanced item collection techniques include inventory systems, item spawning, and conditional collection. These methods provide more depth and complexity, allowing for sophisticated gameplay scenarios.

For example, an inventory system allows players to manage and use collected items strategically, while item spawning can dynamically place items based on game conditions.

\textbf{Practical Applications}

Item collection is essential for

\begin{itemize}
\item
  Implementing collectibles and rewards
\item
  Enhancing player progression and engagement
\item
  Facilitating inventory management and item usage
\item
  Creating dynamic and interactive game environments
\end{itemize}

\textbf{Algorithm Overview}

Let's visualize the algorithm used for item collection:

\begin{enumerate}
\def\labelenumi{\arabic{enumi}.}
\item
  1.
\end{enumerate}

\begin{quote}
\textbf{Initialization}: Set up the player and items within the game scene.

~
\end{quote}

\begin{enumerate}
\def\labelenumi{\arabic{enumi}.}
\setcounter{enumi}{1}
\item
  2.
\end{enumerate}

\begin{quote}
\textbf{Collision Detection}: Continuously check for collisions or interactions between the player and items.

~
\end{quote}

\begin{enumerate}
\def\labelenumi{\arabic{enumi}.}
\setcounter{enumi}{2}
\item
  3.
\end{enumerate}

\begin{quote}
\textbf{Item Collection}: When a collision is detected, collect the item and apply any associated effects (e.g., update inventory, increase score).

~
\end{quote}

\begin{enumerate}
\def\labelenumi{\arabic{enumi}.}
\setcounter{enumi}{3}
\item
  4.
\end{enumerate}

\begin{quote}
\textbf{Visual and Audio Feedback}: Provide feedback to the player through animations, sounds, or UI updates.

~
\end{quote}

\begin{enumerate}
\def\labelenumi{\arabic{enumi}.}
\setcounter{enumi}{4}
\item
  5.
\end{enumerate}

\begin{quote}
\textbf{Conditional Collection}: Implement additional logic for conditional collection, such as requiring a specific item or condition to collect another item.

~
\end{quote}

This algorithm forms the basis for implementing item collection mechanics in Unreal Engine.

\textbf{Reviewing the Code}

The CollectibleItem class is designed to represent a collectible item within the game. This includes handling its initialization, managing its visual representation, and detecting collisions with players or other game elements for item collection.

This section provides an overview of the CollectibleItem class and its key components.

\begin{itemize}
\item
  \textbf{CollectibleItem Class}: Represents a collectible item using a sphere component for collision detection
\item
  \textbf{BeginPlay Function}: Initializes the item when the game starts
\item
  \textbf{Tick Function}: Updates the item's state every frame
\item
  \textbf{CollisionSphere}: A USphereComponent used to detect overlapping events with other actors, such as the player character
\item
  \textbf{OnOverlapBegin Function}: A function that is triggered when another object, like the player, overlaps with the collision sphere, signaling the collection of the item
\end{itemize}

\emph{Class USphereComponent;}

\emph{\#include "CoreMinimal.h"}

\emph{\#include "GameFramework/Actor.h"}

\emph{\#include "CollectibleItem.generated.h"}

UCLASS()

\textbf{class MYGAME\_API} ACollectibleItem : \textbf{public} AActor

\{

~~~~GENERATED\_BODY()

public:

~~~~ACollectibleItem();

protected:

~~~~\textbf{virtual void} BeginPlay() override;

public:

~~~~\textbf{virtual void} Tick(\textbf{float} DeltaTime) override;

private:

~~~~UPROPERTY(VisibleAnywhere)

~~~~USphereComponent* CollisionSphere;

~~~~UFUNCTION()

~~~~\textbf{void} OnOverlapBegin(\textbf{class UPrimitiveComponent}* OverlappedComp, \textbf{class AActor}* OtherActor, \textbf{class UPrimitiveComponent}* OtherComp, int32 OtherBodyIndex, \textbf{bool} bFromSweep, \textbf{const} FHitResult \& SweepResult);

\};

UPROPERTY(EditDefaultsOnly, BlueprintReadOnly, Category = "Effects")

USoundBase* CollectSound;~~// To assign sound effects from Blueprint or Editor

UPROPERTY(EditDefaultsOnly, BlueprintReadOnly, Category = "Effects")

UParticleSystem* CollectEffect;~~// To assign particle effects from Blueprint or Editor

\emph{\textbf{Listing 1-34}}

CollectibleItem.h

This section explains the core functions and components used to implement the collectible item functionality.

\begin{itemize}
\item
  \textbf{ACollectibleItem Constructor}: Initializes the collectible item, creating a sphere component for collision detection and binding the overlap event to the OnOverlapBegin function.
\item
  \textbf{CollisionSphere Initialization}: The USphereComponent is created and assigned as the root component, used to detect overlaps with other actors.
\item
  \textbf{OnOverlapBegin Function}: Triggered when another actor overlaps the sphere, indicating the item has been collected, followed by the destruction of the item.
\end{itemize}

\emph{\#include "CollectibleItem.h"}

\emph{\#include "Components/SphereComponent.h"}

\emph{\#include "GameFramework/Actor.h"}

\emph{\#include "Kismet/GameplayStatics.h"}

ACollectibleItem::ACollectibleItem()

\{

~~~~PrimaryActorTick.bCanEverTick = true;

~~~~CollisionSphere = CreateDefaultSubobject\textless USphereComponent\textgreater(TEXT(\emph{"CollisionSphere"}));

~~~~RootComponent = CollisionSphere;

~~~~CollisionSphere-\textgreater OnComponentBeginOverlap.AddDynamic(\textbf{this}, \&ACollectibleItem::OnOverlapBegin);

\}

\textbf{void} ACollectibleItem::BeginPlay()

\{

~~~~Super::BeginPlay();

\}

\textbf{void} ACollectibleItem::Tick(\textbf{float} DeltaTime)

\{

~~~~Super::Tick(DeltaTime);

\}

// In CollectibleItem.h

private:

~~~~static int32 CollectedItemCount; // Static counter to track total collected items

// In CollectibleItem.cpp

int32 ACollectibleItem::CollectedItemCount = 0; // Initialize static variable

\textbf{void} ACollectibleItem::OnOverlapBegin(UPrimitiveComponent* OverlappedComp, AActor* OtherActor, UPrimitiveComponent* OtherComp, int32 OtherBodyIndex,

bool bFromSweep, const FHitResult \& SweepResult)

\{

~~~~if (OtherActor \&\& (OtherActor != this))

~~~~\{

~~~~~~~~// Increase the count when an item is collected

~~~~~~~~CollectedItemCount++;

~~~~~~~~// Optionally, print the count for debugging

~~~~~~~~UE\_LOG(LogTemp, Log, TEXT("Items Collected: \%d"), CollectedItemCount);

~~~~~~~~// Implement item collection logic here

~~~~~~~~Destroy();

~~~~\}

\}

\emph{\textbf{Listing 1-35}}

CollectibleItem.cpp

\textbf{Implementing Item Effects}

The ApplyItemEffect function is responsible for applying the specific effects of a collectible item (such as increasing the player's score or health) when the item is collected. It also plays an audio cue to provide feedback to the player.

This section explains the ApplyItemEffect function and how it affects the player's interaction with the collectible item:

\begin{itemize}
\item
  \textbf{Sound Binding}: ConstructorHelpers::FObjectFinder\textless USoundBase\textgreater{} is used to load the CollectSound from the specified location.
\item
  \textbf{Particle Effect Binding}: Similar to the sound, ConstructorHelpers::FObjectFinder\textless UParticleSystem\textgreater{} is used to load a particle effect, which is spawned when the item is collected.
\item
  \textbf{Sound and Particle Effect Execution}: In the OnOverlapBegin method, if the sound and particle assets are valid, they are played/spawned when the item is collected.
\end{itemize}

ACollectibleItem::ACollectibleItem()

\{

~~~~PrimaryActorTick.bCanEverTick = true;

~~~~// Initialize collision component

~~~~CollisionSphere = CreateDefaultSubobject\textless USphereComponent\textgreater(TEXT("CollisionSphere"));

~~~~RootComponent = CollisionSphere;

~~~~CollisionSphere-\textgreater OnComponentBeginOverlap.AddDynamic(this, \&ACollectibleItem::OnOverlapBegin);

~~~~// Bind sound and particle system assets through ConstructorHelpers

~~~~static ConstructorHelpers::FObjectFinder\textless USoundBase\textgreater{} SoundAsset(TEXT("SoundWave\textquotesingle/Game/Audio/CollectSound.CollectSound\textquotesingle"));

~~~~if (SoundAsset.Succeeded())

~~~~\{

~~~~~~~~CollectSound = SoundAsset.Object;

~~~~\}

~~~~static ConstructorHelpers::FObjectFinder\textless UParticleSystem\textgreater{} ParticleAsset(TEXT("ParticleSystem\textquotesingle/Game/Particles/CollectEffect.CollectEffect\textquotesingle"));

~~~~if (ParticleAsset.Succeeded())

~~~~\{

~~~~~~~~CollectParticleEffect = ParticleAsset.Object;

~~~~\}

\}

void ACollectibleItem::OnOverlapBegin(UPrimitiveComponent* OverlappedComp, AActor* OtherActor, UPrimitiveComponent* OtherComp, int32 OtherBodyIndex,

~~~~~~~~~~~~~~~~~~~~~~~~~~~~~~~~~~~~~~bool bFromSweep, const FHitResult \& SweepResult)

\{

~~~~if (OtherActor \&\& (OtherActor != this))

~~~~\{

~~~~~~~~// Increase the count when an item is collected

~~~~~~~~CollectedItemCount++;

~~~~~~~~// Optionally, print the count for debugging

~~~~~~~~UE\_LOG(LogTemp, Log, TEXT("Items Collected: \%d"), CollectedItemCount);

~~~~~~~~// Play sound effect on collection

~~~~~~~~if (CollectSound)

~~~~~~~~\{

~~~~~~~~~~~~UGameplayStatics::PlaySoundAtLocation(this, CollectSound, GetActorLocation());

~~~~~~~~\}

~~~~~~~~// Play particle effect on collection

~~~~~~~~if (CollectParticleEffect)

~~~~~~~~\{

~~~~~~~~~~~~UGameplayStatics::SpawnEmitterAtLocation(GetWorld(), CollectParticleEffect, GetActorLocation());

~~~~~~~~\}

~~~~~~~~// Implement item collection logic here

~~~~~~~~Destroy();

~~~~\}

\}

\emph{\textbf{Listing 1-36}}

ApplyItemEffect Function in CollectibleItem.cpp

\textbf{Advanced Item Collection Techniques}

An advanced item collection system involves the introduction of an inventory system to track, store, and manage items collected by the player. This allows the player to collect multiple items and use them at appropriate times, rather than items being immediately consumed or applied.

The implementation of an inventory system includes key components for managing item collection and usage effectively:

\begin{itemize}
\item
  \textbf{InventoryComponent Class}: Manages the inventory system, storing collected items and their usage
\item
  \textbf{AddItem Function}: Adds an item to the inventory and applies its effects or updates the UI
\end{itemize}

\emph{\#pragma once}

class ACollectibleItem;

\emph{\#include "CoreMinimal.h"}

\emph{\#include "Components/ActorComponent.h"}

\emph{\#include "InventoryComponent.generated.h"}

UCLASS(ClassGroup=(Custom), meta=(BlueprintSpawnableComponent))

\textbf{class MYGAME\_API} UInventoryComponent : \textbf{public} UActorComponent

\{

~~~~GENERATED\_BODY()

public:

~~~~UInventoryComponent();

protected:

~~~~\textbf{virtual void} BeginPlay() override;

public:

~~~~\textbf{void} AddItem(ACollectibleItem* Item);

private:

~~~~UPROPERTY(VisibleAnywhere)

~~~~TArray\textless ACollectibleItem*\textgreater{} Items;

\};

\emph{\textbf{Listing 1-37}}

InventoryComponent.h

\emph{\#include "InventoryComponent.h"}

\emph{\#include "CollectibleItem.h"}

UInventoryComponent::UInventoryComponent()

\{

~~~~PrimaryComponentTick.bCanEverTick = false;

\}

\textbf{void} UInventoryComponent::BeginPlay()

\{

~~~~Super::BeginPlay();

\}

\textbf{void} UInventoryComponent::AddItem(ACollectibleItem* Item)

\{

~~~~\textbf{if} (Item)

~~~~\{

~~~~~~~~Items.Add(Item);

~~~~~~~~Item-\textgreater Destroy();

~~~~~~~~\emph{// Implement logic to update inventory UI or apply item effects}

~~~~\}

\}

\emph{\textbf{Listing 1-38}}

InventoryComponent.cpp

This section explored the fundamentals of item collection mechanics in game development. From understanding basic collision detection to implementing advanced techniques like inventory systems, you've gained insights into how item collection enhances gameplay and rewards player interaction.

\textbf{Summary}

As we conclude our exploration of core collision mechanics, we've uncovered the essential building blocks that shape virtual interactions. From blocking movement to triggering events and detecting interactions, these mechanics serve as the invisible framework that governs how objects and characters navigate digital worlds.

Collisions are more than mere technicalities---they define the rules of engagement, setting the stage for meaningful interactions and immersive experiences. Whether it's a door that refuses to open, a platform that carries a player forward, or a simple item pickup that sparks progress, every collision contributes to the flow of gameplay.

In the next chapter, we will dive into advanced collision mechanics, where we move beyond detection and into dynamic reactions. From damage calculation and knockback effects to terrain deformation and visual impact, we'll explore how collisions shape action-packed encounters and bring worlds to life with deeper realism and interactivity.

\chapter{原地转身}
\section{预先处理动画 animation curve}
首先要给动画要添加一些曲线。
\begin{definition}
 Une Matrice est un tableau rectangulaire de la forme
 \begin{equation*}
 A = 
 \left[
 \begin{array}{ccc>{\columncolor{clight2}}c cc} 
 a_{11} & a_{12} & \cdots &  a_{1j} & \cdots & a_{1n}\\ 
 a_{21} & a_{22} & \cdots &  a_{2j} & \cdots & a_{2n}\\ 
 \vdots & \vdots &\vdots & \vdots & \vdots&\vdots \\
 \rowcolor{clight2} a_{i1} & a_{i2} & \cdots & \cellcolor[gray]{.6}a_{ij} & \cdots & a_{in}\\
 \vdots & \vdots &\ddots & \vdots & \vdots &\vdots \\
 a_{m1} & a_{m2} & \cdots & a_{mj} & \cdots & a_{mn}
 \end{array}
 \right]
 \end{equation*}
 où les $a_{ij}$ sont des nombres réels appelés les éléments ou coefficients de la matrice $A$. La matrice précédente est aussi notée par $\left(a_{ij}\right),\;i=1,\cdots,m,\;j=1,\cdots,n$, ou simplement par $\left(a_{ij}\right)$. L'élément $a_{ij}$ est situé à l’intersection de la $i$-ème ligne et de la $j$-ème colonne. Une matrice ayant $m$ lignes et $n$ colonnes
 est appelée une matrice d'ordre $(m,n)$, ou de dimension $m\times n$. Les matrices seront notées habituellement par des lettres capitales $A,\,B,\cdots$, et les éléments par des lettres minuscules $a,\,b,\cdots$.
\end{definition}
\mytitle{SKILLS}{PROBLEM-SOLVING}
On considère les matrices suivantes:
\begin{equation*}
A=
\begin{bmatrix}
1 & -1 & 2\\
3 &  \frac{1}{2} & \sqrt{2}
\end{bmatrix},\;
B=
\begin{bmatrix}
-\sqrt{3} \\
\frac{1}{5}\\
1
\end{bmatrix},\;
C=
\begin{bmatrix}
-1 & 0 & 1 & 5
\end{bmatrix},\;
D=
\begin{bmatrix}
-1 & 1 & 3 \\
5 & 6 & 2 \\
1 & -1 & 0
\end{bmatrix}.
\end{equation*}
\begin{listexos}
 \item La matrice $A$ est de dimension $2\times 3$ et on a $a_{23}=\sqrt{2},\quad a_{13}=2,\quad a_{22}=\frac{1}{2}$.
 \item La matrice $B$ est de dimension $3\times 1$ et on a $b_{11}=-\sqrt{3},\quad b_{21}=\frac{1}{5},\quad b_{31}=1$.
 \item La matrice $C$ est de dimension $1\times 4$ et on a $C_{11}=-1,\quad C_{12}=0,\quad C_{13}=1,\quad C_{14}=5$. 
 \item La matrice $D$ est de dimension $3\times 3$ et on a $D_{33}=0,\quad D_{23}=2,\quad D_{32}=-1$. 
\end{listexos} 
\subsection{小小节矩阵}
\begin{definition}
 Deux matrices $A$ et $B$ sont égales, et on écrit
 $A = B$, si elles ont même dimension et si leurs éléments correspondants sont égaux.
\end{definition}
\mytitle{技能}{解题过程}
考虑如下矩阵 $
A=
\begin{bmatrix}
0 & 1  \\
2 & 8 
\end{bmatrix},
\;
B=
\begin{bmatrix}
0 & x^{2}\\
\sqrt{y} & z 
\end{bmatrix}$.
\begin{itemize}
 \item Trouver $x,\,y\in\mathbb{R}$ tels que $A=B$.
\end{itemize}
\vspace{1cm}
\begin{mybox}{解题过程}
 Trouver les valeurs possibles de $x,\,y\in\mathbb{R}$ telles que les matrices
 \[ 
 E=
 \begin{bmatrix}
 x^{2}-5 & 1 \\
 3 & -2y-4  
 \end{bmatrix}
 \quad
 F=
 \begin{bmatrix}
 1 & 1 \\
 x+2 & 0  
 \end{bmatrix}.
 \]
 soit égales.
\end{mybox}
\section{Matrices particulières}
Dans cette on va définir quelques matrices spéciales.
\subsection{Matrice ligne}
\begin{definition}
 Une matrice ligne est une matrice comportant une seule ligne. Une matrice ligne a donc
 pour dimension $1\times n$. Une matrice ligne a la forme suivante:
 \begin{equation*}
 A=
 \begin{bmatrix}
 a_{11} & a_{12} & \cdots & a_{1n}
 \end{bmatrix}_{1\times n}.
 \end{equation*}
\end{definition}
\mytitle{SKILLS}{PROBLEM-SOLVING}
La matrice
$
A=
\begin{bmatrix}
1 & 2 & -1 & 5 & 0
\end{bmatrix}
$ est une matrice ligne de dimension $1\times 5$.
\subsection{Matrice colone}
\begin{definition}
 Une matrice colonne est une matrice comportant une seule colonne.  Une matrice colonne a donc
 pour dimension $m\times 1$. Une matrice colonne a la forme suivante:
 \begin{equation*}
 A=\begin{bmatrix}
 a_{11}  \\
 a_{21}  \\
 \vdots  \\
 a_{n1}
 \end{bmatrix}_{n\times 1}.
 \end{equation*}
\end{definition}
\mytitle{SKILLS}{PROBLEM-SOLVING}
La matrice
$
A=
\begin{bmatrix}
-1 \\
0 \\
2 \\
4
\end{bmatrix}$ est une matrice colonne de dimension $4\times 1$.
\section{Opérations sur les matrice}
Dans cette section on va définir les opérations algébriques sur les matrices.
\subsection{Somme de deux matrices}
\begin{definition}
 Soit $A$ et $B$ deux matrices de même dimension. La somme de $A$ et $B$, écrite $A+B$, est la matrice obtenue en ajoutant les éléments correspondants des deux matrices.
 \begin{equation*}
 \begin{aligned}
 &\text{Si}\quad A=\begin{bmatrix}
 a_{11} & a_{12} & a_{13}\\
 a_{21} & a_{22}  & a_{23}
 \end{bmatrix}_{2\times 3}\;\text{et}\quad
 B=\begin{bmatrix}
 b_{11} & b_{12} & b_{13}\\
 b_{21} & b_{22}  & b_{23}
 \end{bmatrix}_{2\times 3},
 \\ \ \\
 &\text{alors}\quad A+B=\begin{bmatrix}
 a_{11}+b_{11} & a_{12}+b_{12} & a_{13}+b_{13}\\
 a_{21}+b_{21} & a_{22}+b_{22}  & a_{23}+b_{23}
 \end{bmatrix}_{2\times 3}.
 \end{aligned}
 \end{equation*}
\end{definition}
\mytitle{SKILLS}{PROBLEM-SOLVING}
On considère les matrices: 
\begin{equation*}
A=
\begin{bmatrix}
-2 & 0 & 1\\
1 & 3  & -1\\
\end{bmatrix},\;
B=
\begin{bmatrix}
1 & 2 \\
1 & 0 \\
0 & 4
\end{bmatrix},\;
C=
\begin{bmatrix}
-2 & 1 \\
4 & 3 \\
-1 & -1
\end{bmatrix},\;
D=
\begin{bmatrix}
1 & 1 & 0\\
2 & 0 & 4
\end{bmatrix}.
\end{equation*}
\begin{listexos}
 \item Calchler $B+C$.
 \begin{equation*}
 B+C=\begin{bmatrix}
 1 & 2 \\
 1 & 0 \\
 0 & 4
 \end{bmatrix}+
 \begin{bmatrix}
 -2 & 1 \\
 4 & 3 \\
 -1 & -1
 \end{bmatrix}=
 \begin{bmatrix}
 1+(-2) & 2+1 \\
 1+4 & 0+3 \\
 0+1 & 4+(-1)
 \end{bmatrix}=
 \begin{bmatrix}
 -1 & 3 \\
 5 & 3 \\
 1 & 3
 \end{bmatrix}.
 \end{equation*} 
 \item Calculer $A+D$.
 \begin{equation*}
 A+D=\begin{bmatrix}
 -2 & 0 & 1 \\
 1 & 3  & -1
 \end{bmatrix}+
 \begin{bmatrix}
 1 & 1 & 0\\
 2 & 0 & 4
 \end{bmatrix}=
 \begin{bmatrix}
 -2+1 & 0+1 & 1+0\\
 1+2 & 3+0 & -1+4
 \end{bmatrix}=
 \begin{bmatrix}
 -1 & 1 & 1\\
 3 & 3 & 3
 \end{bmatrix}.
 \end{equation*}
 \item  La somme de $A$ et $B$ n’est pas définie car $A$ et $B$ ne sont pas de même dimension.
\end{listexos}
\newpage
\section{Exercices}
\begin{multicols*}{2} 
 \sect{Calcul matriciel}
 \begin{Exercice}
 On considère les matrices
 \[ 
 A=\begin{pmatrix}
 4 & 8 \\
 1 & 2  
 \end{pmatrix},\;
 B=\begin{pmatrix}
 3 & 9 \\
 1 & 1  
 \end{pmatrix}.
 \]
 \begin{listexos}
 \item Calculer $A+B,\,AB,\;BA,\,A^2$ et $B^2$.
 \item A-t-on $(A+B)^2=A^2+2AB+B^2$? 
 \item Mêmes questions pour les matrices 
 \[ 
 A=\begin{pmatrix}
 1 & 0 \\
 2 & 1 
 \end{pmatrix},\;B=\begin{pmatrix}
 2 & 0 \\
 1 & 2  
 \end{pmatrix}.
 \]
 \end{listexos}
 \end{Exercice}
 \begin{Exercice}
 On considère les matrices
 \[ 
 A=\begin{pmatrix}
 x & 5 \\
 0 & 2x  
 \end{pmatrix},\;B=\begin{pmatrix}
 y & 7 \\
 -1 & 3y  
 \end{pmatrix}.
 \] 
 \begin{listexos}
 \item  Trouver $x,y\in\mathbb{R}$ tels que
 \[ 
 A+B=\begin{pmatrix}
 4 & 12 \\
 -1 & 17  
 \end{pmatrix}.
 \]
 \item Trouver $x,y\in\mathbb{R}$ tels que
 \[ 
 2A-4B=\begin{pmatrix}
 -5 & -18 \\
 4 & -16  
 \end{pmatrix}.
 \]
 \end{listexos}
 \end{Exercice}  
 \begin{Exercice}
 On considère la matrice
 \[ 
 A=
 \begin{pmatrix}
 x & 1 \\
 2 & 3  
 \end{pmatrix},\quad x\in\mathbb{R}.
 \]
 \begin{itemize}
 \item Trouver $x$ tel que $A^2=\begin{pmatrix}
 6 & 1 \\
 2 & 11  
 \end{pmatrix}$ 
 \end{itemize}
 \end{Exercice}  
 \begin{Exercice}
 On considère les matrices 
 \[ 
 \begin{aligned}
 &A=
 \begin{pmatrix}
 1 & 3 \\
 -4 & 2 \\
 0 & 7 
 \end{pmatrix},
 \;
 B=
 \begin{pmatrix}
 -2 & 0 \\
 -2 & 1 \\
 8 & 1 
 \end{pmatrix},
 \\
 &C=
 \begin{pmatrix}
 -4 & 6 \\
 -14 & 7 \\
 24 & 17 
 \end{pmatrix}.
 \end{aligned}
 \]
 \begin{itemize}
 \item Trouver $x, y\in\mathbb{R}$ tels que $xA+yB=C$.
 \end{itemize}
 \end{Exercice}  
 \begin{Exercice}
 Calculer si possible les produits matriciels suivants: 
 \begin{listexos}
 \item 
 \[ 
 \begin{pmatrix}
 2 & 5 \\
 3 & 6\\
 4 & 7 
 \end{pmatrix}
 \times
 \begin{pmatrix}
 2 & 5 \\
 4 & 6 
 \end{pmatrix}
 \]
 \item 
 \[ 
 \begin{pmatrix}
 -1 & 4 & 5
 \end{pmatrix}
 \times
 \begin{pmatrix}
 0 & 1 & 6\\
 3 & -1 & 4\\
 3 & 5 & -2
 \end{pmatrix}
 \]
 \item 
 \[ 
 \begin{pmatrix}
 2 & -3 & 4\\
 -1 & 2& 6\\
 4 & -3 & -3
 \end{pmatrix}
 \times
 \begin{pmatrix}
 1 & 2 & -3\\
 0 & -4 & 1
 \end{pmatrix}
 ^T
 \]
 \end{listexos}
 \end{Exercice}  
\end{multicols*}
\chapter{强制将数据和逻辑分离}
强制将数据和逻辑分离
\begin{itemize}
	\item 简单: 只需在 UAnimInstance 子类中重载一个函数,无需创建和管理额外的代理(Proxy)结构体。
	\item 快速实现: 对于简单的属性复制和计算,这种方式的代码量更少,更容易理解。
\end{itemize}
\newpage
上述代码会在实例化策略为\texttt{InstancedPerActor}时,调用\texttt{CreateNewInstanceOfAbility}创建技能实例:

\section{创建}
\subsection{技能组件创建}
\subsection{初始化}
对于GAS的初始化,要注意\texttt{ACharacter}的两个函数:
一个用于服务端初始化,一个用于客户端初始化。
\newpage
上述代码会在实例化策略为\texttt{InstancedPerActor}时,调用\texttt{CreateNewInstanceOfAbility}创建技能实例:

\section{创建}
\subsection{技能组件创建}
\subsection{初始化}
对于GAS的初始化,要注意\texttt{ACharacter}的两个函数:
一个用于服务端初始化,一个用于客户端初始化。
\newpage
\pagestyle{empty}
测试水平和垂直文字排列。
\begin{tikzpicture}[overlay,remember picture]
		 % 在页面正中心绘制一个半径为 2cm 的圆
		\draw[thick,fill =blue!25!white,draw=yellow] (current page.center) circle [radius=2cm];
		\node[fill=gray,inner sep=0pt,rectangle] at(current page.center){\parbox{18cm}{\centering\textcolor{red}{\bfseries\scshape 这里显示水平文字}}};
		%
		\node[fill=gray,inner sep=0pt,rectangle] at(current page.center){\rotatebox{-90}{\parbox{18cm}{\centering\textcolor{red}{\bfseries\scshape 这里显示水平文字}}}};
		%
		\path[thick,fill =blue!25!white,draw=yellow] (current page.east) circle [radius=5mm];
		\node[fill=cyan,inner sep=0pt,rectangle]at ($ (current page.east)+(-1cm,0mm) $)
		{
			\rotatebox{-90}{\parbox{\paperheight}{\centering\textcolor{red}{\bfseries\scshape \leftmark}}}
		};%奇数页
		%
		\path[bottom color=white,top color=\BoxColor] (current page.north west) rectangle ($(current page.south west)+(15mm,0mm)$);		
		\node[inner sep=0pt,rectangle](leftbanner)at ($ (current page.west)+(7mm,0mm) $)
		{
			\rotatebox{90}{\centering\textcolor{red}{\bfseries\scshape \leftmark}}
		};
	
\end{tikzpicture}
\end{document}